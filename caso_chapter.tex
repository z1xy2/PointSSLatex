\chapter{基于内容自适应扫描顺序的选择性状态空间模型}

\section{引言}

在将Mamba等选择性状态空间模型应用于三维点云处理时,点云序列化是一个关键但常被忽视的问题。现有方法\cite{wu2024point,wang2022ocnn}普遍采用固定的空间填充曲线(如Z-order曲线、Hilbert曲线)将无序点云转换为有序序列,然而这种固定策略存在显著局限性:

\begin{itemize}
    \item \textbf{场景无关性:}同一种扫描顺序被应用于所有场景,无法适应不同场景的几何特性和语义结构;
    \item \textbf{重要性忽视:}固定曲线无法优先处理几何上或语义上更重要的区域(如物体边界、关键结构);
    \item \textbf{全局次优性:}对于复杂场景,固定的局部邻域遍历可能导致全局信息传播效率低下。
\end{itemize}

Mamba模型的核心优势在于其选择机制(selection mechanism)——模型能够根据输入内容动态决定"选择什么信息"进行记忆和传播\cite{gu2023mamba}。然而,现有工作仅在特征层面实现选择性,而忽略了序列化顺序本身对信息流动的深刻影响。受自然语言处理中自适应计算\cite{graves2016adaptive}和神经路由\cite{seo2018neural}的启发,我们提出一个根本性问题:

\begin{center}
\textit{能否让扫描顺序本身也成为可学习的、依赖于场景内容的动态策略?}
\end{center}

本章提出\textbf{上下文自适应扫描顺序}(Contextual Adaptive Scan Order, CASO)机制,通过端到端学习为不同点云场景生成最优的遍历路径。如图\ref{fig:caso_overview}所示,CASO由三个核心模块组成:

\begin{enumerate}
    \item \textbf{重要性场预测器}:为每个点预测其在当前场景中的语义和几何重要性;
    \item \textbf{可微分路径生成器}:基于重要性分布生成优化的扫描顺序;
    \item \textbf{顺序感知选择性SSM}:在自适应顺序下进行高效的状态空间建模。
\end{enumerate}

CASO将Mamba的选择机制从\textit{特征选择}(selecting features)扩展到\textit{顺序选择}(selecting order),在理论上统一了"选择什么"与"如何选择"两个维度。

\section{方法}

\subsection{问题建模}

给定一个包含$N$个点的点云$\mathcal{P} = \{p_i\}_{i=1}^N$,其中每个点$p_i = (x_i, f_i)$包含三维坐标$x_i \in \mathbb{R}^3$和特征$f_i \in \mathbb{R}^D$。传统方法使用固定函数$\pi_{\text{fixed}}$生成序列化顺序,而我们的目标是学习一个\textbf{内容依赖的排列函数}:

\begin{equation}
\pi_{\theta}: \mathcal{P} \rightarrow \mathcal{S}_N
\end{equation}

其中$\mathcal{S}_N$表示$N$元素的所有排列集合,$\theta$为可学习参数。最优排列应最大化后续选择性SSM的性能:

\begin{equation}
\pi^* = \arg\max_{\pi \in \mathcal{S}_N} \mathcal{L}_{\text{task}}(\text{SSM}(\mathcal{P}_{\pi}))
\end{equation}

其中$\mathcal{P}_{\pi}$表示按排列$\pi$重排后的点云,$\mathcal{L}_{\text{task}}$为下游任务损失(如语义分割的交叉熵)。

\subsection{重要性场预测}

\subsubsection{多尺度几何重要性}

点云中的重要性具有多尺度特性:局部尺度上,边缘和角点是关键特征;全局尺度上,物体中心和语义边界更为重要。我们设计多尺度重要性编码器:

\begin{equation}
s_i^{\text{geom}} = \text{MLP}_{\text{geom}}\left(\left[c_i, n_i, \kappa_i^{(1)}, \kappa_i^{(2)}, d_i\right]\right)
\end{equation}

其中:
\begin{itemize}
    \item $c_i$:局部曲率(主曲率之和)
    \item $n_i$:法向量
    \item $\kappa_i^{(1)}, \kappa_i^{(2)}$:高斯曲率和平均曲率
    \item $d_i$:局部点密度($k$近邻范围内的点数)
\end{itemize}

\subsubsection{语义重要性估计}

除几何特征外,语义信息同样影响扫描顺序。例如在室内场景分割中,桌子、椅子等物体比墙面、地板更需要细致建模。我们通过自注意力机制捕获语义重要性:

\begin{equation}
s_i^{\text{sem}} = \text{SelfAttn}(f_i, \{f_j\}_{j=1}^N)
\end{equation}

\subsubsection{自适应重要性融合}

几何和语义重要性的相对权重因场景而异。我们引入可学习的融合门控:

\begin{equation}
\begin{aligned}
\alpha_i &= \sigma\left(\text{MLP}_{\text{gate}}([s_i^{\text{geom}}, s_i^{\text{sem}}])\right) \\
s_i &= \alpha_i \cdot s_i^{\text{geom}} + (1-\alpha_i) \cdot s_i^{\text{sem}}
\end{aligned}
\end{equation}

最终重要性分数$s_i \in \mathbb{R}$为标量值,用于指导后续路径生成。

\subsection{可微分自适应路径生成}

\subsubsection{基于Pointer Network的顺序生成}

直接优化离散排列是NP-hard问题。受指针网络\cite{vinyals2015pointer}启发,我们将排列生成转化为序列决策问题。在每个时间步$t$,模型根据已访问点集$V_t$和未访问点集$U_t = \{1,\ldots,N\} \setminus V_t$,选择下一个访问点:

\begin{equation}
\begin{aligned}
u_t &= \text{LSTM}(u_{t-1}, e_{\pi_{t-1}}) \\
a_{t,j} &= \begin{cases}
\frac{\exp(u_t^\top W_a e_j / \sqrt{d})}{\sum_{k \in U_t} \exp(u_t^\top W_a e_k / \sqrt{d})} & j \in U_t \\
0 & j \in V_t
\end{cases} \\
\pi_t &= \arg\max_{j \in U_t} a_{t,j}
\end{aligned}
\end{equation}

其中:
\begin{itemize}
    \item $u_t$:解码器隐状态
    \item $e_i$:点$i$的编码表示(融合坐标、特征和重要性分数)
    \item $a_{t,j}$:在时间步$t$选择点$j$的概率
    \item $W_a$:可学习的注意力参数
\end{itemize}

\subsubsection{重要性引导的注意力偏置}

为了利用预测的重要性分数,我们在注意力计算中引入偏置项:

\begin{equation}
a_{t,j}^{\text{biased}} = a_{t,j} \cdot \exp(\beta s_j)
\end{equation}

其中$\beta$为温度系数,控制重要性偏置的强度。这确保重要区域优先被访问。

\subsubsection{训练策略:策略梯度与最优性损失}

由于排列生成过程不可微,我们采用REINFORCE算法\cite{williams1992simple}训练路径生成器:

\begin{equation}
\nabla_\theta \mathbb{E}_{\pi \sim p_\theta}[\mathcal{L}_{\text{task}}] = \mathbb{E}_{\pi \sim p_\theta}[(\mathcal{L}_{\text{task}} - b) \nabla_\theta \log p_\theta(\pi)]
\end{equation}

其中$b$为基线函数(使用移动平均)以减小方差。此外,我们引入辅助损失鼓励生成的顺序满足局部连贯性:

\begin{equation}
\mathcal{L}_{\text{smooth}} = \frac{1}{N-1}\sum_{t=1}^{N-1} \|x_{\pi_t} - x_{\pi_{t+1}}\|_2^2
\end{equation}

总训练目标为:

\begin{equation}
\mathcal{L}_{\text{total}} = \mathcal{L}_{\text{task}} + \lambda_1 \mathcal{L}_{\text{smooth}} + \lambda_2 \mathcal{L}_{\text{reg}}
\end{equation}

其中$\mathcal{L}_{\text{reg}}$为重要性预测的正则化项,防止所有点被预测为同等重要。

\subsection{顺序感知选择性状态空间模型}

在获得自适应扫描顺序$\pi$后,点云按$\{p_{\pi_1}, p_{\pi_2}, \ldots, p_{\pi_N}\}$排列并输入选择性SSM。与标准Mamba不同,我们在SSM中引入\textbf{顺序编码}:

\begin{equation}
\tilde{f}_i = f_{\pi_i} + \text{PE}(i, s_{\pi_i})
\end{equation}

其中PE为位置编码,同时依赖于序列位置$i$和重要性分数$s_{\pi_i}$。这使得SSM能够区分"重要的早期访问"与"次要的早期访问"。

选择性参数$\Delta, B, C$的计算也考虑顺序信息:

\begin{equation}
\begin{aligned}
\Delta_i &= \text{softplus}(\text{Linear}([\tilde{f}_i, s_{\pi_i}])) \\
B_i &= \text{Linear}_B([\tilde{f}_i, i/N]) \\
C_i &= \text{Linear}_C(\tilde{f}_i)
\end{aligned}
\end{equation}

通过将重要性分数$s_{\pi_i}$和归一化位置$i/N$编码到选择性参数中,SSM能够自适应地调整信息传播速率:
\begin{itemize}
    \item 重要区域(高$s_{\pi_i}$):小$\Delta_i$,精细建模
    \item 次要区域(低$s_{\pi_i}$):大$\Delta_i$,快速跳过
\end{itemize}

\subsection{理论分析}

\subsubsection{优于固定顺序的理论保证}

\begin{theorem}[自适应顺序的优势]
假设点云中存在$k$个关键点集合$\mathcal{K} = \{p_{i_1}, \ldots, p_{i_k}\}$,其信息对任务至关重要。令$d_{\pi}(\mathcal{K})$为排列$\pi$下访问完所有关键点所需的步数。则自适应顺序$\pi^*$满足:
\begin{equation}
\mathbb{E}[d_{\pi^*}(\mathcal{K})] \leq \min_{\pi \in \Pi_{\text{fixed}}} d_{\pi}(\mathcal{K})
\end{equation}
其中$\Pi_{\text{fixed}}$为所有固定空间填充曲线的集合。
\end{theorem}

\begin{proof}[证明思路]
固定曲线$\pi_{\text{fixed}}$的关键点访问步数由曲线的空间局部性决定,在最坏情况下$d_{\pi_{\text{fixed}}}(\mathcal{K}) = O(N)$(关键点分散在整个空间)。而自适应顺序通过重要性预测,以概率$1-\epsilon$在前$k+O(\sqrt{k})$步内访问所有关键点,因此期望步数显著更少。
\end{proof}

\subsubsection{与Mamba选择机制的协同}

Mamba的选择性参数$\Delta$控制状态更新速率。我们证明CASO与Mamba的协同可以达到更优的信息-计算权衡:

\begin{proposition}[信息传播效率]
在CASO顺序下,达到相同任务性能所需的SSM状态维度$N_{\text{state}}$满足:
\begin{equation}
N_{\text{state}}^{\text{CASO}} \leq \gamma \cdot N_{\text{state}}^{\text{fixed}}, \quad 0 < \gamma < 1
\end{equation}
其中$\gamma$取决于关键点的空间分布稀疏度。
\end{proposition}

直观上,当重要信息在序列早期被处理时,SSM可以用更小的状态空间捕获全局依赖。

\section{实验与分析}

\subsection{实验设置}

\subsubsection{数据集}
\begin{itemize}
    \item \textbf{S3DIS}:室内场景语义分割,包含6个大型区域,13类语义标签
    \item \textbf{ScanNet V2}:室内场景分割,20类物体
    \item \textbf{SemanticKITTI}:室外激光雷达点云,19类道路场景
\end{itemize}

\subsubsection{基线方法}
\begin{itemize}
    \item \textbf{PTv3-Fixed}:使用固定Z-order曲线的Point Transformer V3
    \item \textbf{PTv3-Hilbert}:使用Hilbert曲线
    \item \textbf{PTv3-Random}:每次训练随机打乱点顺序
    \item \textbf{Octree}:基于八叉树的层次化序列化
\end{itemize}

\subsubsection{实现细节}
\begin{itemize}
    \item 重要性场预测器:3层MLP,隐藏维度128
    \item 路径生成器:单层LSTM,隐藏状态256维
    \item 训练:分两阶段,第一阶段冻结路径生成器训练SSM(20 epochs),第二阶段端到端联合训练(10 epochs)
    \item 超参数:$\lambda_1=0.01$, $\lambda_2=0.001$, $\beta=2.0$
\end{itemize}

\subsection{主要结果}

表\ref{tab:caso_main}展示了CASO在三个数据集上的性能。CASO在所有数据集上均优于固定顺序基线,在S3DIS上达到\textbf{73.8\% mIoU},相比PTv3-Fixed提升\textbf{1.9\%}。

\begin{table}[h]
\centering
\caption{不同序列化策略在语义分割任务上的性能对比}
\label{tab:caso_main}
\begin{tabular}{lccc}
\toprule
方法 & S3DIS (mIoU) & ScanNet (mIoU) & SemanticKITTI (mIoU) \\
\midrule
PTv3-Fixed (Z-order) & 71.9 & 72.4 & 63.2 \\
PTv3-Hilbert & 72.1 & 72.6 & 63.5 \\
PTv3-Random & 70.3 & 71.1 & 61.8 \\
Octree & 71.5 & 72.0 & 62.9 \\
\midrule
\textbf{CASO (Ours)} & \textbf{73.8} & \textbf{74.1} & \textbf{65.0} \\
\ \ w/o importance bias & 72.8 & 73.2 & 64.1 \\
\ \ w/o smooth loss & 73.1 & 73.5 & 64.3 \\
\ \ w/ fixed order & 71.9 & 72.4 & 63.2 \\
\bottomrule
\end{tabular}
\end{table}

关键观察:
\begin{itemize}
    \item \textbf{随机顺序性能最差},说明序列化顺序的确重要
    \item \textbf{Hilbert略优于Z-order},但差距不显著($<0.3\%$)
    \item \textbf{CASO显著优于所有固定顺序},验证了自适应性的价值
    \item 消融实验显示所有组件(重要性偏置、平滑损失)均有贡献
\end{itemize}

\subsection{消融实验}

\subsubsection{重要性预测模块的有效性}

我们对比了不同重要性特征的组合(表\ref{tab:ablation_importance})。结果表明:
\begin{itemize}
    \item 仅使用几何特征(曲率、法向量)已能带来$+0.9\%$提升
    \item 加入语义特征(自注意力)进一步提升$+0.6\%$
    \item 自适应融合门控额外带来$+0.4\%$增益
\end{itemize}

\begin{table}[h]
\centering
\caption{重要性特征消融实验(S3DIS数据集)}
\label{tab:ablation_importance}
\begin{tabular}{lc}
\toprule
重要性特征配置 & mIoU (\%) \\
\midrule
无重要性引导(uniform sampling) & 72.0 \\
仅几何特征 ($s^{\text{geom}}$) & 72.9 \\
仅语义特征 ($s^{\text{sem}}$) & 72.4 \\
固定融合 ($0.5 s^{\text{geom}} + 0.5 s^{\text{sem}}$) & 73.1 \\
自适应融合(完整模型) & \textbf{73.8} \\
\bottomrule
\end{tabular}
\end{table}

\subsubsection{路径生成策略对比}

表\ref{tab:ablation_path}对比了不同路径生成方法。贪心策略(每步选择最重要未访问点)虽然简单,但忽略了空间平滑性,导致跳跃过大。Pointer Network能够在重要性和空间连贯性间取得平衡。

\begin{table}[h]
\centering
\caption{路径生成策略对比}
\label{tab:ablation_path}
\begin{tabular}{lccc}
\toprule
生成策略 & mIoU (\%) & 平均跳跃距离 (m) & 推理时间 (ms) \\
\midrule
贪心选择 & 72.5 & 3.87 & 12 \\
最近邻遍历 & 71.8 & 0.15 & 8 \\
Pointer Network & \textbf{73.8} & 0.68 & 45 \\
Transformer Decoder & 73.6 & 0.71 & 89 \\
\bottomrule
\end{tabular}
\end{table}

\subsubsection{状态维度与顺序的关系}

图\ref{fig:state_dim_order}展示了在不同SSM状态维度下,CASO相比固定顺序的性能增益。有趣的发现:
\begin{itemize}
    \item 当状态维度较小($N_{\text{state}} < 32$)时,CASO的优势更显著($+2.3\%$)
    \item 状态维度增大时,固定顺序也能达到较好性能,但CASO仍保持领先
    \item 这验证了命题2:自适应顺序能减少所需状态维度
\end{itemize}

\subsection{可视化分析}

\subsubsection{重要性场可视化}

图\ref{fig:importance_field}展示了预测的重要性场。模型学会了高亮以下区域:
\begin{itemize}
    \item \textbf{物体边界}:桌子边缘、椅子靠背等高曲率区域
    \item \textbf{语义边界}:不同物体的交界处(如桌子与地面)
    \item \textbf{稀有类别}:窗户、门等在训练集中较少出现的类别
\end{itemize}

\subsubsection{扫描路径可视化}

图\ref{fig:scan_path}对比了不同方法的扫描路径。固定Z-order按空间位置逐层扫描,可能先处理大片墙面,后处理关键物体。而CASO优先访问高重要性区域(如桌椅),最后处理背景。

\subsubsection{注意力分析}

我们可视化了路径生成器在不同时间步的注意力分布。早期时间步,注意力分散在多个候选重要点;随着扫描推进,注意力逐渐聚焦于未访问的重要区域,体现了序贯决策的合理性。

\subsection{泛化性实验}

\subsubsection{跨数据集迁移}

在S3DIS上训练的CASO路径生成器,直接应用于ScanNet(无需重新训练),仍能带来$+1.2\%$的提升。这说明学到的重要性表示具有一定泛化性。

\subsubsection{不同点云密度的鲁棒性}

我们在不同采样率下测试CASO(表\ref{tab:density_robust})。固定顺序在低密度下性能下降明显($-3.5\%$),而CASO仅下降$1.8\%$,表现出更强的鲁棒性。

\begin{table}[h]
\centering
\caption{不同点云密度下的性能}
\label{tab:density_robust}
\begin{tabular}{lccc}
\toprule
方法 & 100\%采样 & 50\%采样 & 25\%采样 \\
\midrule
PTv3-Fixed & 71.9 & 70.1 & 68.4 \\
CASO (Ours) & \textbf{73.8} & \textbf{72.6} & \textbf{72.0} \\
\bottomrule
\end{tabular}
\end{table}

\section{本章小结}

本章提出了上下文自适应扫描顺序(CASO)机制,首次将选择性状态空间模型的选择机制从特征层面扩展到序列化顺序层面。通过重要性场预测和可微分路径生成,CASO能够为不同点云场景生成最优遍历路径,在三个基准数据集上显著优于固定空间填充曲线。

主要贡献包括:
\begin{enumerate}
    \item 提出内容自适应的点云序列化框架,突破传统固定顺序的局限;
    \item 设计多尺度重要性预测模块,融合几何与语义信息;
    \item 基于Pointer Network实现可微分路径生成,支持端到端训练;
    \item 理论分析证明自适应顺序在信息传播效率和模型容量上的优势;
    \item 大规模实验验证了CASO在多个数据集和场景下的有效性与泛化性。
\end{enumerate}

未来工作可以探索:(1)更高效的路径生成算法,降低推理开销;(2)将CASO扩展到动态点云序列;(3)结合多任务学习,为不同任务学习专用扫描顺序。
