\chapter{基于多路径动态路由的自适应序列化网络}

第三章提出的GGAM通过显式注入几何先验增强了特征表示,有效缓解了Mamba序列化过程中的几何信息丢失问题。第四章提出的ASD-SSM通过自适应尺度解耦实现了层次化的局部多尺度特征学习,在给定序列化顺序下显著提升了模型对细粒度局部模式的建模能力。然而,无论是GGAM还是ASD-SSM,都基于PTv3提供的固定序列化顺序(如Z-order或Hilbert曲线)进行特征学习。这种固定策略将同一种扫描顺序应用于所有场景和所有区域,无法根据不同场景的几何特性和语义结构进行自适应调整,从而限制了模型的整体性能。

为解决上述问题,本章提出\textbf{多路径自适应序列化网络}(Multi-path Adaptive Serialization Network, MASA),从全局序列化策略的角度与ASD-SSM形成互补。MASA的完整处理流程如\cref{fig:masa}所示。MASA与ASD-SSM的功能定位明确不同:ASD-SSM关注"在给定序列化下如何更好地学习局部多尺度特征",解决的是\textit{局部尺度建模}问题;MASA关注"如何选择更好的序列化以建立全局连接",解决的是\textit{全局路径选择}问题。MASA通过几何条件化的动态路由机制,根据场景的局部几何特性自适应选择或组合序列化路径,使得ASD-SSM能够在更优的点云序列化下进行局部特征学习,从而形成"全局序列化优化(MASA) + 局部尺度解耦(ASD-SSM)"的双层互补架构。



具体而言,MASA预定义$K=5$种互补的序列化策略(包括4种空间填充曲线和1种语义驱动序列化),并根据局部几何特性动态分配路由权重,将Mamba的选择机制从\textit{特征选择}(selecting features)扩展到\textit{路径选择}(selecting paths)。这种动态路由机制能够为不同几何特性的区域选择最适合的序列化方式:对于规则平面区域使用高效的空间填充曲线,对于复杂物体边界使用重要性优先的语义序列化,从而为ASD-SSM的局部多尺度学习提供更优的全局上下文。后续将围绕方法设计和相关消融实验进行详细介绍。

\section{问题分析与动机}

\subsection{ASD-SSM的局限与MASA的互补定位}

第四章提出的ASD-SSM通过自适应尺度解耦机制,在局部邻域内实现了多尺度特征学习,能够有效捕获点云的细粒度几何结构。然而,ASD-SSM的设计聚焦于"在给定点云序列化下,如何更好地学习局部多尺度特征",其核心假设是输入点云已经按照某种固定顺序(如Z-order或Hilbert曲线)排列。这一假设带来两个根本性限制:

\textbf{限制1:序列化顺序的固定性}。ASD-SSM依赖PTv3提供的固定序列化策略,该策略对所有场景、所有区域采用相同的空间扫描顺序。然而,不同几何结构适合的序列化方式存在显著差异:规则平面区域(如地板、墙面)适合Z-order等空间填充曲线,能够高效维持空间邻近性;而复杂物体边界(如桌椅边缘、物体交界处)则更需要语义感知的序列化来优先处理关键特征。固定序列化无法根据局部几何特性进行自适应调整,限制了ASD-SSM的局部特征学习效果。

\textbf{限制2:局部建模与全局路径的脱节}。ASD-SSM通过自适应尺度解耦在序列化邻域内进行局部多尺度学习,但序列化顺序本身决定了哪些点会被视为"邻近"。不合适的序列化可能导致语义相关但空间分离的点(如同一物体的不同部分)难以在局部邻域内建立连接,从而降低ASD-SSM的多尺度特征学习质量。例如,在处理细长结构(如柱子、桌腿)时,固定的空间填充曲线可能将结构的两端分离到序列的不同位置,使得ASD-SSM难以在局部窗口内同时观察到完整的结构信息。

基于上述分析,MASA的设计动机在于:为ASD-SSM提供更优的全局序列化策略,使其局部多尺度学习能够在更合理的点云顺序下进行。MASA与ASD-SSM形成明确的功能分工:MASA负责\textit{全局层面}的序列化路径选择,根据场景几何特性动态组合多种序列化策略,建立合理的全局点云顺序;ASD-SSM负责\textit{局部层面}的多尺度特征解耦,在MASA提供的序列化下进行细粒度的局部特征学习。两者共同构成"全局序列化优化 + 局部尺度建模"的双层架构,相互补充、缺一不可。

\subsection{问题形式化}

给定包含$N$个点的点云$\mathcal{P} = \{p_1, p_2, \ldots, p_N\}$,传统方法使用固定序列化顺序(如Z-order或Hilbert曲线),将点云映射为一维序列。然而,固定序列化无法适应不同场景的几何特性。直接学习任意排列则面临组合爆炸($N!$种可能)和训练困难等问题。

为此,我们提出多路径路由方案:从$K$种预定义序列化策略$\{\pi_1, \pi_2, \ldots, \pi_K\}$中动态选择或组合。对于点云$\mathcal{P}$,每种策略$\pi_k$定义一个排列函数,将点映射到序列位置。MASA学习路由权重$w = [w_1, w_2, \ldots, w_K]$,其中$w_k \geq 0$且$\sum_{k=1}^K w_k = 1$,实现软路由融合:
\begin{equation}
h = \sum_{k=1}^K w_k \cdot h^{(k)}
\end{equation}
其中$h^{(k)}$为在序列化$\pi_k$下通过Mamba处理后的特征。这种方式完全可微分,支持端到端训练,复杂度为$O(KN\log N)$,其中$K$为常数。

\section{方法设计}

MASA的核心思想是通过动态路由机制自适应选择或组合多种序列化策略。整体方法包括三个关键模块:(1)几何感知的重要性场预测,为动态路由提供几何先验;(2)多路径序列化策略设计,提供互补的序列化候选;(3)几何条件化动态路由网络,根据局部几何特性分配序列化权重。接下来将依次介绍各模块的设计细节。

\subsection{几何感知的重要性场预测}
\label{sec:importance}

重要性场预测为动态路由和语义序列化提供几何先验。我们设计轻量级的重要性编码器,融合多尺度几何特征,为每个点预测其在场景中的重要性分数。

\subsubsection{局部几何描述符提取}

对于每个点$p_i$,在其$k$-近邻$\mathcal{N}_k(p_i)$上计算局部几何特征。受经典点云几何分析方法\cite{pauly2003shape}启发,我们通过主成分分析(PCA)提取以下几何描述符:

\textbf{(1)曲率特征}:通过计算协方差矩阵$C_i$的特征值$\lambda_1 \geq \lambda_2 \geq \lambda_3$,定义三种几何基元:
\begin{equation}
\begin{aligned}
\text{linearity:} \quad & l_i = (\lambda_1 - \lambda_2) / \lambda_1 \\
\text{planarity:} \quad & p_i = (\lambda_2 - \lambda_3) / \lambda_1 \\
\text{scattering:} \quad & s_i = \lambda_3 / \lambda_1
\end{aligned}
\end{equation}
其中$l_i$接近1表示点位于线状结构(如边缘),$p_i$接近1表示点位于平面,$s_i$接近1表示点云散乱分布。

\textbf{(2)法向变化}:计算局部法向量的方差$v_i = \text{Var}(\{n_j\}_{j \in \mathcal{N}_k(i)})$,反映表面平滑度。高方差区域通常对应物体边界或结构突变。

\textbf{(3)点云密度}:定义为$d_i = k / V_i$,其中$V_i$为$k$-近邻的外接球体积。密度变化可反映场景的采样均匀性和结构复杂度。

组合几何描述符为$g_i = [l_i, p_i, s_i, v_i, d_i] \in \mathbb{R}^5$。

\subsubsection{多尺度重要性编码}

为了捕获不同尺度的几何模式,采用多尺度编码器。在不同的邻域半径$r \in \{r_1, r_2, r_3\}$下分别提取几何描述符$g_i^{(r)}$,然后通过多层感知机进行编码:
\begin{equation}
\begin{aligned}
h_i^{(r)} &= \text{MLP}_r([f_i, g_i^{(r)}]), \quad r \in \{r_1, r_2, r_3\} \\
s_i &= \text{Sigmoid}(\text{MLP}_{\text{agg}}([h_i^{(r_1)}, h_i^{(r_2)}, h_i^{(r_3)}]))
\end{aligned}
\end{equation}
其中$r \in \{0.05, 0.1, 0.2\}$(单位为米)表示不同的邻域半径,$f_i$为点的输入特征。最终重要性分数$s_i \in [0,1]$通过Sigmoid激活归一化。高重要性分数表示该点位于几何结构的关键位置(如物体边界、角点、结构突变处)。

\subsection{多路径序列化策略设计}

我们设计$K=5$种互补的序列化策略,分别捕获不同的空间和语义特性。与从头设计所有策略不同,我们采用更高效的方案:利用Point Transformer V3已有的4种空间序列化作为基础路径,并新增1种语义驱动的序列化作为核心创新。

\subsubsection{空间填充曲线序列化}

Point Transformer V3的序列化模块已计算了以下4种空间填充曲线\cite{ptv3},我们直接利用这些预计算结果,无需额外计算开销:

\textbf{(1)Z-order曲线(xyz轴顺序)}:通过Morton编码实现高效的空间索引,适合规则分布的点云。对于归一化坐标$(x,y,z) \in [0,1]^3$,Morton码定义为:
\begin{equation}
m(p_i) = \text{Interleave}(\lfloor 2^{10} x_i \rfloor, \lfloor 2^{10} y_i \rfloor, \lfloor 2^{10} z_i \rfloor)
\end{equation}
排列$\pi_{\text{z-xyz}}$按Morton码升序排列所有点。Z-order曲线通过递归二进制空间划分,在局部尺度下具有良好的空间聚集性。

\textbf{(2)Z-order曲线(yxz轴顺序)}:与(1)使用相同的Morton编码原理,但改变坐标交织顺序为$(y,x,z)$,获得不同的空间遍历路径。这种变体在不同方向上的空间局部性有所差异,可为模型提供互补的邻域信息。

\textbf{(3)Hilbert曲线(xyz轴顺序)}:相比Z-order曲线,Hilbert曲线具有更好的空间连续性保持特性,在序列化过程中能更好地维持几何结构的整体连贯性。我们采用3D Hilbert曲线的快速近似算法\cite{lawder2000using}进行计算。

\textbf{(4)Hilbert曲线(yxz轴顺序)}:类似(2),通过改变轴顺序$(y,x,z)$获得Hilbert曲线的变体,增强路径多样性。

这4种空间序列化已由Point Transformer V3预先计算并存储在点云数据结构中,MASA可直接使用而无需额外计算开销。

\subsubsection{重要性引导的语义序列化}

为了克服纯空间序列化的语义无关性,我们提出\textbf{重要性引导的广度优先搜索}(Importance-Guided BFS)作为第5条路径。该方法平衡了语义重要性和几何连续性,是MASA的核心贡献之一。

\textbf{(5)重要性引导BFS序列化}:该算法结合了以下设计原则:
\begin{itemize}
\item \textbf{多起点策略}:选择top-$K_s$个重要点作为种子,实现快速覆盖关键区域
\item \textbf{分层扩展}:每层基于几何邻近性扩展,保持空间连续性
\item \textbf{加权转移}:综合考虑几何距离和重要性分数,优先访问关键区域
\item \textbf{邻居候选构建}:利用上述4种空间序列化,为每个点提取前后各$k$个邻居,形成丰富的候选集
\end{itemize}

具体地,对于当前点$p_i$到候选邻居点$p_j$的转移概率定义为:
\begin{equation}
P(i \to j) \propto \exp\left(-\frac{\|x_i - x_j\|^2}{2\sigma^2}\right) \cdot (s_j + \epsilon)
\end{equation}
其中$x_i, x_j$为点坐标,$s_j$为候选点的重要性分数的重要性编码器预测),$\sigma$为距离尺度参数(实验中设为0.1),$\epsilon$为平滑项(设为0.01)。该转移概率的第一项保证了几何上的连续性,避免大幅度的空间跳跃;第二项则确保重要区域(如物体边界、角点)优先被访问,从而在序列前部获得更多注意力。

算法流程如下:首先根据重要性分数选择top-$K_s$个点作为初始种子,将它们加入序列并标记为已访问;然后对每个种子点,根据转移概率$P(i \to j)$选择下一个候选点加入序列;重复此过程直到所有点被访问。这种序列化方式确保重要区域(如物体边界)优先被处理,同时保持几何上的连续性,避免了纯重要性排序导致的空间跳跃问题。

\subsection{几何条件化动态路由网络}

动态路由网络是MASA的核心创新,负责根据局部几何特性为每个空间区域分配序列化权重。与为每个点独立分配权重不同,我们采用体素化分块路由策略,既保持了空间连贯性,又显著减少了路由决策的数量。

\subsubsection{基于体素网格的分块路由}

直接为每个点分配独立的路由权重会导致过度碎片化和计算不稳定。我们采用\textbf{体素化分块路由}(voxel-based patch-wise routing)策略:

\textbf{(1)空间体素化}:将点云空间划分为大小为$v \times v \times v$的体素网格,其中$v$为体素边长(如$0.5$m)。每个非空体素形成一个空间块$\mathcal{B}_m$:
\begin{equation}
\mathcal{B}_m = \{p_i \mid \lfloor x_i / v \rfloor = (g_x^m, g_y^m, g_z^m)\}
\end{equation}
其中$(g_x^m, g_y^m, g_z^m)$为块$m$的体素网格坐标。这种体素化将连续的3D空间离散化为规则网格,使得空间邻近的点自然聚集到同一体素块内。

\textbf{(2)分块序列化邻域}:对于每个块$\mathcal{B}_m$内的点,按序列化顺序$\pi_k$重排并进行padding,形成大小为$K_b$的标准化序列化邻域(实验中设$K_b=16$):
\begin{equation}
\begin{aligned}
\text{order}_k &= \text{argsort}(\{\text{code}_k(p_i)\}_{i \in \mathcal{B}_m}) \\
\mathcal{B}_m^{(k)} &= \{\text{pad}(p_{\text{order}_k(j)})\}_{j=1}^{K_b}
\end{aligned}
\end{equation}
其中$\text{code}_k(p_i)$表示点$p_i$在序列化策略$\pi_k$下的编码值(如Morton码或Hilbert码)。padding操作确保每个块都包含恰好$K_b$个点:若块内点数不足$K_b$,则重复最后几个点;若超过$K_b$,则分割为多个子块。这种标准化使得后续的Mamba处理可以采用固定的序列长度。

\textbf{(3)块级路由权重}:每个块$\mathcal{B}_m$共享相同的路由权重$w^{(m)} \in \Delta^K$(其中$\Delta^K$表示$K$维单纯形),这既保持了空间连贯性,又显著减少了路由决策的数量(从$N$个点减少到$M \ll N$个块)。相邻体素块的路由权重可能不同,从而实现局部自适应的序列化策略选择。

\subsubsection{路由权重计算}

对于每个块$\mathcal{B}_m$,首先通过池化聚合块内特征,得到块级表示:
\begin{equation}
\begin{aligned}
\bar{f}_m &= \text{MaxPool}(\{f_i\}_{i \in \mathcal{B}_m}) \\
\bar{g}_m &= \text{MeanPool}(\{g_i\}_{i \in \mathcal{B}_m}) \\
\bar{s}_m &= \text{MeanPool}(\{s_i\}_{i \in \mathcal{B}_m})
\end{aligned}
\end{equation}
其中$f_i$为点的输入特征,$g_i$为几何描述符,$s_i$为重要性分数。使用最大池化聚合输入特征可以保留块内的显著性信息,而使用平均池化聚合几何和重要性信息则可以获得块的整体几何特性。

然后通过门控网络计算路由权重:
\begin{equation}
\begin{aligned}
z_m &= \text{MLP}_{\text{gate}}([\bar{f}_m, \bar{g}_m, \bar{s}_m]) \in \mathbb{R}^K \\
w_m &= \text{Softmax}(z_m / \tau)
\end{aligned}
\end{equation}
其中$\text{MLP}_{\text{gate}}$为2-3层多层感知机,$\tau$为温度参数,控制路由的尖锐程度。训练初期使用较大的$\tau$(如1.0)保持探索性,使各路径权重分布较为均匀;训练后期退火至0.5使决策更加确定,让模型逐渐收敛到最优路径组合。

\subsubsection{多路径特征聚合}

获得路由权重后,我们对$K$条路径的序列化特征进行加权融合。整个过程包括以下步骤:

\textbf{(1)路径序列化}:对于每条路径$k$,按序列化顺序$\pi_k$重排点云。具体地,根据该路径的编码函数(Morton码、Hilbert码或importance-guided BFS),为所有点计算序列位置,然后按位置升序排列。

\textbf{(2)标准化邻域构建}:按照描述的分块策略,将重排后的点云划分为大小为$K_b$的标准化邻域。每个邻域内的点在该路径的序列化下是连续的。

\textbf{(3)共享Mamba处理}:所有路径使用同一个Mamba块进行处理:
\begin{equation}
h^{(k)} = \text{Mamba}_{\text{shared}}(\text{neighborhoods}_k)
\end{equation}
这里的关键设计在于:所有路径在Mamba层面共享参数,仅序列化顺序不同。因此计算开销相比独立路径方案大幅降低(仅为独立方案的$1/K$),同时保持了路径多样性带来的性能增益。

\textbf{(4)软路由融合}:根据路由权重$w_m$对不同路径的输出进行加权融合:
\begin{equation}
h = \sum_{k=1}^K w_k \cdot h^{(k)}
\end{equation}

\textbf{(5)逆序列化映射}:通过inverse索引将融合后的特征恢复到原始点云顺序,得到最终输出。

\subsubsection{负载均衡正则化}

为防止路由坍塌(即所有块都选择同一路径),我们引入负载均衡损失。该损失鼓励各路径的平均使用率接近均匀分布:
\begin{equation}
\mathcal{L}_{\text{balance}} = \sum_{k=1}^K \left(\frac{1}{M}\sum_{m=1}^M w_m^{(k)} - \frac{1}{K}\right)^2
\end{equation}
其中$M$为总块数,$w_m^{(k)}$为块$m$在路径$k$上的权重。该损失项惩罚偏离均匀分布的路径使用情况,确保所有预定义路径都能得到有效利用,从而保持路径多样性。

\subsection{层级自适应的MASA配置}

在深层神经网络中,不同stage处理的特征具有不同的感受野和抽象程度。我们设计了层级自适应配置,在不同stage使用不同的体素化粒度和路由策略,以适应特征的层次性。

\textbf{递增体素粒度}:随着网络深度增加,特征的感受野逐渐扩大,因此采用递增策略调整体素大小:
\begin{equation}
v_s = v_0 \cdot \alpha^s
\end{equation}
其中$v_0$为初始体素大小(实验中设为$0.5$m),$\alpha$为增长因子(设为$1.5$),$s$为stage索引。这使得浅层关注局部几何细节,深层关注全局语义结构。例如,在5-stage网络中,体素大小依次为$0.5$m、$0.75$m、$1.125$m、$1.69$m、$2.53$m。

\textbf{选择性启用MASA}:并非所有stage都需要动态路由。实验发现前2个stage启用MASA效果最佳,因为浅层特征对序列化顺序更敏感,动态路由带来显著增益($+1.3\%$ mIoU);而深层stage可选择性关闭MASA,因为深层特征已高度抽象,固定序列化即可满足需求,关闭MASA可减少计算开销($-15\%$推理时间),性能损失可忽略($<0.2\%$)。这种选择性配置在性能和效率之间取得了良好平衡。

\subsection{训练策略}

MASA采用端到端训练。在基础模型的训练框架下,我们仅为MASA模块引入负载均衡损失作为额外的正则项:
\begin{equation}
\mathcal{L}_{\text{MASA}} = \mathcal{L}_{\text{base}} + \lambda_{\text{bal}} \mathcal{L}_{\text{balance}}
\end{equation}
其中$\mathcal{L}_{\text{base}}$为基础模型的损失函数(交叉熵损失,继承自PTv3),$\mathcal{L}_{\text{balance}}$为负载均衡损失(防止路由坍塌),权重系数设置为$\lambda_{\text{bal}} = 0.01$。

MASA特有的路由温度参数$\tau$采用简单的余弦退火策略,从$\tau_{\max} = 1.0$退火至$\tau_{\min} = 0.3$:
\begin{equation}
\tau(t) = \tau_{\min} + \frac{1}{2}(\tau_{\max} - \tau_{\min})(1 + \cos(\pi t / T))
\end{equation}
其中$t$为当前epoch,$T$为总epoch数。初期较高的温度使各路径权重分布较为均匀,保持探索性;后期较低的温度使路由决策更加确定,帮助模型收敛到最优配置。

其他训练配置与基础模型保持一致:使用AdamW优化器\cite{loshchilov2017adamw},初始学习率$6 \times 10^{-3}$,采用OneCycleLR学习率调度策略。总训练时长为3000 epochs,批量大小为6,启用混合精度训练以减少显存占用。所有实验在4张Nvidia A6000 GPU上进行。

\section{实验设置与分析}

\subsection{实验设置}

\subsubsection{数据集}

我们在三个基准数据集上进行评估,以验证MASA在不同场景类型和数据规模下的泛化能力:

\textbf{(1)S3DIS}\cite{armeni20163d}:室内场景语义分割数据集,包含6个大型区域(来自3栋建筑),覆盖271个房间,总计约273百万个点。标注包含13个语义类别(如天花板、地板、墙、桌子、椅子等)。我们采用Area 5作为测试集,其余区域作为训练集的标准划分方式。

\textbf{(2)ScanNet V2}\cite{dai2017scannet}:大规模室内场景分割数据集,包含1513个训练场景和312个验证场景,标注20类常见室内物体。该数据集的场景更加多样化,包含不同房间类型和物体布局,对模型的泛化能力要求更高。

\textbf{(3)SemanticKITTI}\cite{behley2019semantickitti}:户外激光雷达点云数据集,包含43552帧,标注19类道路场景元素(如道路、车辆、行人、建筑等)。我们使用序列00-07和09-10作为训练集,序列08作为验证集。该数据集的特点是场景规模大、点云稀疏,与室内数据集形成互补。

\subsubsection{基线方法}

我们对比了以下基线方法,以验证MASA动态路由机制的有效性:

\textbf{(1)PTv3-Fixed}:使用固定Z-order曲线的Point Transformer V3\cite{ptv3},这是最常用的序列化策略。

\textbf{(2)PTv3-Hilbert}:使用Hilbert曲线替代Z-order,验证不同空间填充曲线的影响。

\textbf{(3)PTv3-Random}:每次训练时随机打乱点顺序,作为下界基线。

\textbf{(4)PTv3-Multi}:在不同层使用不同固定序列化(如\cite{wu2024point}中的多顺序轮换策略),但不使用动态路由。

\subsubsection{实现细节}

我们在Point Transformer V3基础上实现MASA模块。网络采用5-stage编码-解码架构,编码器通道数为(32, 64, 128, 256, 512),解码器通道数为(64, 64, 128, 256)。每个stage包含多个Point Transformer块和Mamba块,具体配置与PTv3保持一致。

MASA配置:路径数$K=5$(4条空间序列化来自PTv3,1条importance-guided BFS为新增),邻域大小$K_b=16$,初始体素大小$v_0=0.5$m,增长因子$\alpha=1.5$。动态路由仅在前2个stage启用,以平衡性能和效率。

重要性预测器采用3尺度设计($r \in \{0.05, 0.1, 0.2\}$m),每个尺度使用2层MLP(隐藏维度128)进行编码,聚合后通过3层MLP(隐藏维度256)输出重要性分数。路由网络采用3层MLP(隐藏维度256),输出$K$维logits后通过温度softmax得到路由权重。

训练使用AdamW优化器,初始学习率$6 \times 10^{-3}$,权重衰减0.05,采用OneCycleLR调度。总训练3000 epochs,批量大小6,启用混合精度训练。损失权重设置为$\lambda_{\text{bal}}=0.01$和$\lambda_{\text{reg}}=0.0001$。路由温度从1.0退火至0.3。所有实验在4张Nvidia A6000 GPU上进行,单次训练约需40-45小时。

\subsection{主要结果}

\cref{tab:masa_main}展示了MASA在三个数据集上的性能。MASA在所有数据集上均显著优于固定顺序基线:在S3DIS上达到\textbf{73.2\% mIoU},相比PTv3-Fixed提升\textbf{1.3\%};在ScanNet上达到\textbf{73.6\% mIoU},提升\textbf{1.2\%};在SemanticKITTI上达到\textbf{64.7\% mIoU},提升\textbf{1.5\%}。

\begin{table}[htbp!]
\centering
\caption{不同序列化策略在语义分割任务上的性能对比}
\label{tab:masa_main}
\begin{tabular}{lccc}
\toprule
方法 & S3DIS (mIoU) & ScanNet (mIoU) & SemanticKITTI (mIoU) \\
\midrule
PTv3-Fixed (Z-order) & 71.9 & 72.4 & 63.2 \\
PTv3-Hilbert & 72.1 & 72.6 & 63.5 \\
PTv3-Random & 70.3 & 71.1 & 61.8 \\
PTv3-Multi & 72.3 & 72.8 & 63.8 \\
\midrule
\textbf{MASA (Ours)} & \textbf{73.2} & \textbf{73.6} & \textbf{64.7} \\
\ \ w/o dynamic routing & 72.5 & 72.9 & 63.9 \\
\ \ w/o importance & 72.7 & 73.1 & 64.2 \\
\ \ w/ hard routing & 72.8 & 73.2 & 64.3 \\
\bottomrule
\end{tabular}
\end{table}

从实验结果可以得到以下关键观察:

\textbf{(1)序列化顺序的重要性}:PTv3-Random的性能显著低于其他方法(低1.6-1.4\%),说明序列化顺序对模型性能至关重要。随机顺序破坏了点云的空间邻近性,使得Mamba难以建立有效的长程依赖。

\textbf{(2)不同空间填充曲线的差异}:PTv3-Hilbert略优于PTv3-Fixed(Z-order),但差距不显著($<0.3\%$)。这表明不同空间填充曲线各有优劣,单一曲线难以适应所有场景类型。

\textbf{(3)静态多路径的局限}:PTv3-Multi通过在不同层使用不同固定顺序获得一定增益(+0.4\%),但提升有限。这是因为静态策略无法根据具体场景进行自适应调整。

\textbf{(4)动态路由的有效性}:MASA显著优于所有固定顺序基线(+0.9-1.5\%),验证了动态路由机制的价值。特别是在SemanticKITTI上的提升更大(+1.5\%),这可能是因为户外场景的几何复杂度更高,更需要自适应的序列化策略。

\textbf{(5)各组件的贡献}:消融实验显示,去除动态路由(w/o dynamic routing)导致性能下降0.7-0.8\%,去除重要性场预测(w/o importance)下降0.5\%,使用硬路由替代软路由(w/ hard routing)下降0.4\%。这表明所有组件(动态路由、重要性预测、软路由融合)均有独立贡献。

\subsection{消融实验}

为了深入理解MASA各组件的作用机制,我们设计了系统的消融实验。所有消融实验均在S3DIS数据集上进行,以保证结果的可比性。

\subsubsection{序列化策略的贡献}

\cref{tab:ablation_paths}分析了不同序列化策略组合的效果。实验结果表明,仅使用PTv3的4种空间序列化(无动态路由)并通过静态融合(均匀加权)达到72.4\% mIoU,相比单一Z-order(71.9\%)提升0.5\%。这说明多种空间序列化的简单集成已能带来一定的性能增益。

\begin{table}[htbp!]
\centering
\caption{序列化策略组合消融实验(S3DIS数据集)}
\label{tab:ablation_paths}
\begin{tabular}{lcc}
\toprule
策略组合 & 动态路由 & mIoU (\%) \\
\midrule
仅Z-order xyz(PTv3基线) & - & 71.9 \\
PTv3 4种空间序列化 & 静态融合 & 72.4 \\
+ importance-BFS & 静态融合 & 72.7 \\
\textbf{MASA完整(K=5)} & \textbf{动态路由} & \textbf{73.2} \\
\bottomrule
\end{tabular}
\end{table}

加入新设计的importance-guided BFS语义序列化后,性能进一步提升至72.7\%(+0.3\%)。这验证了语义驱动序列化的价值:通过优先访问重要区域,模型能够更好地捕获关键几何结构。最后,引入动态路由机制后,MASA达到73.2\% mIoU,相比静态融合额外提升0.5\%。这表明几何条件化的路径选择能够根据局部特性自适应调整序列化策略,优于简单的均匀加权。

综合来看,多路径序列化的性能提升来自三个层面:(1)空间序列化的多样性(+0.5\%),(2)语义序列化的引入(+0.3\%),(3)动态路由的自适应性(+0.5\%)。三者共同贡献了1.3\%的总提升。

\subsubsection{路由策略对比}

\cref{tab:routing_strategies}对比了不同路由方式的性能与效率。均匀加权(所有路径权重相等)相当于简单集成,性能为72.5\% mIoU。硬路由(每个块只选择一条路径,即top-1路径)虽然计算更高效(推理时间减少26\%,内存减少16\%),但性能略低0.4\%。这是因为硬路由丢失了多路径融合带来的鲁棒性。

\begin{table}[htbp!]
\centering
\caption{路由策略对比}
\label{tab:routing_strategies}
\begin{tabular}{lccc}
\toprule
路由策略 & mIoU (\%) & 推理时间 (ms) & 内存 (GB) \\
\midrule
均匀加权 & 72.5 & 38 & 4.2 \\
硬路由(Top-1) & 72.8 & 31 & 3.8 \\
\textbf{软路由(MASA)} & \textbf{73.2} & \textbf{42} & \textbf{4.5} \\
\bottomrule
\end{tabular}
\end{table}

MASA采用的软路由(动态加权融合)取得了最佳性能(73.2\%),代价是略高的计算开销(+11\%推理时间,+7\%内存)。这种性能-效率权衡是合理的:软路由通过加权融合多条路径的输出,能够综合利用不同序列化的优势,获得更加鲁棒的特征表示。在实际应用中,可根据场景需求选择合适的路由策略。

\subsubsection{MASA与ASD-SSM的互补性验证}

为了验证MASA与ASD-SSM的互补关系,我们设计了系统的消融实验,分别测试单独使用各模块和组合使用的性能。\cref{tab:masa_asdssm_complementary}展示了实验结果。

\begin{table}[htbp!]
\centering
\caption{MASA与ASD-SSM互补性消融实验(S3DIS数据集)}
\label{tab:masa_asdssm_complementary}
\begin{tabular}{lcc}
\toprule
模型配置 & mIoU (\%) & 说明 \\
\midrule
Baseline (PTv3) & 71.9 & 固定Z-order序列化,无多尺度解耦 \\
+ ASD-SSM only & 73.5 & 局部多尺度学习,固定序列化 \\
+ MASA only & 73.2 & 动态序列化,无多尺度解耦 \\
\midrule
\textbf{+ ASD-SSM + MASA} & \textbf{75.2} & 完整模型 \\
\bottomrule
\end{tabular}
\end{table}

实验结果表明,单独使用ASD-SSM相比Baseline提升1.6\%,单独使用MASA提升1.3\%,证明两个模块都有独立价值。更重要的是,同时使用ASD-SSM和MASA达到75.2\% mIoU,相比单独使用进一步提升1.7-2.0\%,且超过两者简单相加的预期($71.9\% + 1.6\% + 1.3\% = 74.8\%$)。这表明两者存在正向协同作用。

具体而言,MASA为ASD-SSM提供了更优的全局序列化,使得局部邻域内包含更多语义相关的点,从而提升多尺度特征解耦的质量。例如,在处理细长结构(如桌腿)时,importance-guided BFS能够将结构的不同部分保持在较近的序列位置,使得ASD-SSM的局部窗口能够同时观察到完整的结构信息。反过来,ASD-SSM的多尺度特征学习能力使得MASA的路由网络能够更准确地识别不同几何结构,从而做出更好的路径选择决策。

这一消融实验充分证明了MASA与ASD-SSM的互补关系:两者解决不同层面的问题(全局序列化 vs 局部尺度建模),功能定位明确,协同作用显著,共同构成了PointSS的核心技术体系。

\subsection{可视化分析}

为了直观理解MASA学习到的路由分布,我们在S3DIS测试场景上可视化了不同区域的路径选择情况。\cref{fig:routing_weights}展示了代表性场景的路由权重分布。

从可视化结果可以观察到明显的模式:

\textbf{(1)平面区域偏好空间填充曲线}:地面、墙面等大平面区域主要使用Z-order和Hilbert曲线(占比$>65\%$)。这是因为这些区域的几何结构简单且规则,空间填充曲线能够高效维持空间邻近性。

\textbf{(2)复杂物体依赖语义序列化}:桌椅、门窗等复杂物体更多依赖importance-guided BFS序列化(占比$\sim 40\%$)。这些物体具有复杂的几何边界和结构突变,语义驱动的序列化能够优先处理关键特征,提升识别准确率。

\textbf{(3)边缘区域的均匀分布}:物体边缘和角点展现出更均匀的路径分布,各路径权重相近。这表明模型通过多路径融合捕获多角度信息,以应对边缘区域的高几何复杂度和不确定性。

\textbf{(4)空间连贯性}:相邻体素块的路由权重分布具有一定的连续性,不会出现剧烈跳变。这验证了体素化分块路由策略的合理性,保持了空间上的连贯性。

这些可视化结果验证了MASA的设计直觉:不同几何特性的区域确实需要不同的序列化策略,而动态路由机制能够自动学习这种对应关系。

\subsection{计算效率分析}

\cref{tab:efficiency}对比了不同方法的计算开销。MASA相比PTv3-Fixed增加约10\%的训练时间和8\%的推理时间,内存占用增加约10\%。这些额外开销主要来自:(1)重要性场预测(~3\%时间),(2)多路径Mamba处理(~5\%时间,但通过参数共享已大幅降低),(3)动态路由网络(~2\%时间)。

\begin{table}[htbp!]
\centering
\caption{计算效率对比(S3DIS数据集,单场景)}
\label{tab:efficiency}
\begin{tabular}{lccc}
\toprule
方法 & 训练时间 (s/iter) & 推理时间 (ms) & GPU内存 (GB) \\
\midrule
PTv3-Fixed & 0.42 & 39 & 4.1 \\
PTv3-Multi & 0.45 & 41 & 4.3 \\
\textbf{MASA (Ours)} & \textbf{0.46} & \textbf{42} & \textbf{4.5} \\
\bottomrule
\end{tabular}
\end{table}

考虑到MASA带来的性能提升(+1.3\% mIoU),这种效率-性能权衡是可接受的。特别是在推理阶段,仅增加8\%的时间开销即可获得显著的性能提升,在实际应用中具有良好的实用性。此外,通过选择性启用MASA(仅在前2个stage),我们在保持性能的同时进一步降低了计算开销。

值得注意的是,MASA的计算复杂度为$O(KN\log N)$,其中$K=5$为常数。虽然理论上比单一序列化($O(N\log N)$)高5倍,但由于参数共享和高效的实现,实际开销远小于理论值。这得益于以下优化:(1)4种空间序列化复用PTv3预计算结果,无额外开销;(2)所有路径共享Mamba参数,避免了重复计算;(3)体素化分块大幅减少了路由决策次数。

\section{本章小结}

本章针对现有方法在序列化过程中采用固定策略的局限性,提出了多路径自适应序列化网络(MASA)。MASA从全局序列化策略的角度与第四章的ASD-SSM形成互补,共同构成PointSS的"全局序列化优化 + 局部尺度建模"双层架构。

MASA的核心贡献包括三个方面:首先,设计了5种互补的序列化策略,包括4种空间填充曲线(复用PTv3预计算结果)和1种新提出的importance-guided BFS语义序列化。语义序列化通过优先访问几何重要区域,克服了纯空间序列化的语义无关性。其次,提出了几何条件化的动态路由机制,通过体素化分块路由和门控网络,根据局部几何特性自适应分配序列化权重。该机制保持了空间连贯性,同时显著减少了路由决策的数量。第三,通过共享参数的多路径处理和层级自适应配置,在保持线性时间复杂度的同时实现了序列化的场景自适应性。

实验结果表明,MASA在S3DIS、ScanNet、SemanticKITTI三个数据集上分别达到73.2\%、73.6\%、64.7\% mIoU,相比固定序列化提升1.2\%-1.5\%。可视化分析显示,MASA能够自动学习不同区域的最优序列化策略:平面区域主要使用空间填充曲线(占比$>65\%$),复杂物体更依赖语义序列化(占比$\sim 40\%$),边缘区域则通过多路径融合获得鲁棒性。

与ASD-SSM协同使用时,PointSS在S3DIS上达到75.2\% mIoU,相比单独使用MASA或ASD-SSM均有显著提升(+2.0\%和+1.7\%),且超过两者简单相加的预期。这验证了两者的互补性:MASA提供更优的全局序列化使得ASD-SSM能够在更合理的局部邻域内进行多尺度学习,而ASD-SSM的多尺度特征则帮助MASA做出更准确的路径选择决策。两者共同构成了PointSS处理点云数据的核心机制,为后续任务提供了丰富的几何和语义信息。