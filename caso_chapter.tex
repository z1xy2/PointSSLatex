\chapter{基于多路径动态路由的自适应序列化网络}

\section{引言}

在将Mamba等选择性状态空间模型应用于三维点云处理时,点云序列化是一个关键但常被忽视的问题。现有方法\cite{wu2024point,wang2022ocnn}普遍采用固定的空间填充曲线(如Z-order曲线、Hilbert曲线)将无序点云转换为有序序列,然而这种固定策略存在显著局限性:(1)\textbf{场景无关性}——同一种扫描顺序被应用于所有场景,无法适应不同场景的几何特性和语义结构;(2)\textbf{重要性忽视}——固定曲线无法优先处理几何上或语义上更重要的区域(如物体边界、关键结构);(3)\textbf{局部性受限}——不同几何结构(如平面、曲面、边缘)适合不同的序列化方式,固定策略无法兼顾。

Mamba模型的核心优势在于其选择机制(selection mechanism)——模型能够根据输入内容动态决定"选择什么信息"进行记忆和传播\cite{gu2023mamba}。然而,现有工作仅在特征层面实现选择性,而忽略了序列化顺序本身对信息流动的深刻影响。受自然语言处理中混合专家模型(Mixture of Experts)\cite{shazeer2017outrageously}和动态神经路由\cite{rosenbaum2018routing}的启发,我们提出一个根本性问题:能否让模型根据局部几何特性,\textbf{动态选择}最适合的序列化路径?

本章提出\textbf{多路径自适应序列化网络}(Multi-path Adaptive Serialization Network, MASA),通过几何条件化的动态路由机制实现内容自适应的点云序列化。如图\ref{fig:masa_overview}所示,MASA由三个核心模块组成:(1)\textbf{几何感知的重要性场预测},提取多尺度几何特征用于路由决策;(2)\textbf{多路径序列化生成器},预定义多种互补的序列化策略;(3)\textbf{几何条件化动态路由},根据局部几何特征自适应选择或组合序列化路径。MASA将Mamba的选择机制从\textit{特征选择}(selecting features)扩展到\textit{路径选择}(selecting paths),在保持计算效率的同时实现序列化的自适应性。

\section{方法}

\subsection{问题建模与动机}

给定一个包含$N$个点的点云$\mathcal{P} = \{p_i\}_{i=1}^N$,其中每个点$p_i = (x_i, f_i)$包含三维坐标$x_i \in \mathbb{R}^3$和特征$f_i \in \mathbb{R}^D$。传统方法使用固定函数$\pi_{\text{fixed}}$生成序列化顺序,而直接学习任意排列函数$\pi: \mathcal{P} \rightarrow \mathcal{S}_N$(其中$\mathcal{S}_N$为$N$元素的所有排列集合)面临两大挑战:

\textbf{挑战1:组合爆炸}。对于$N$个点,可能的排列数为$N!$,直接优化离散排列是NP-hard问题。即使采用Pointer Network等序列生成方法,计算复杂度仍为$O(N^2)$,对于大规模点云(如$N > 10^4$)难以实际应用。

\textbf{挑战2:训练困难}。排列生成的离散性使得难以端到端训练,需要依赖REINFORCE等策略梯度方法,训练过程方差大、不稳定。

为此,我们提出\textbf{多路径路由}的替代方案:不直接生成完整排列,而是从$K$种预定义的序列化策略集合$\{\pi_1, \pi_2, \ldots, \pi_K\}$中进行动态选择或组合。优化目标转化为:

\begin{equation}
w^* = \arg\max_{w \in \Delta^K} \mathcal{L}_{\text{task}}\left(\text{SSM}\left(\sum_{k=1}^K w_k \mathcal{P}_{\pi_k}\right)\right)
\end{equation}

其中$w = [w_1, \ldots, w_K]$为路由权重,$\Delta^K$表示$K$维概率单纯形。这种软路由(soft routing)方式完全可微分,支持端到端训练,同时将复杂度降低至$O(KN\log N)$,其中$K$为常数(通常$K=4\sim 6$)。

\subsection{多路径序列化策略设计}

我们设计$K=5$种互补的序列化策略,分别捕获不同的空间和几何特性:

\subsubsection{空间填充曲线}

\textbf{(1) Z-order曲线}:通过Morton编码实现高效的空间索引,适合规则分布的点云。对于归一化坐标$(x,y,z) \in [0,1]^3$,Morton码定义为:
\begin{equation}
m(p_i) = \text{Interleave}(\lfloor 2^{10} x_i \rfloor, \lfloor 2^{10} y_i \rfloor, \lfloor 2^{10} z_i \rfloor)
\end{equation}
排列$\pi_{\text{z}}$按Morton码升序排列所有点。

\textbf{(2) Hilbert曲线}:相比Z-order具有更好的空间局部性保持特性,适合连续曲面。我们采用3D Hilbert曲线的快速近似算法\cite{lawder2000using}。

\subsubsection{几何驱动序列化}

\textbf{(3) 重要性优先序列化}:基于预测的重要性分数$s_i$(见第\ref{sec:importance}节)降序排列:
\begin{equation}
\pi_{\text{imp}}(i) < \pi_{\text{imp}}(j) \Leftrightarrow s_i > s_j
\end{equation}
这确保几何上的关键区域(如边缘、角点)优先被处理。

\textbf{(4) 密度自适应序列化}:在高密度区域采用精细遍历,低密度区域快速跳过。具体地,计算每个点的局部密度$d_i = \|\mathcal{N}_k(p_i)\|$($k$-近邻数量),然后采用密度聚类引导的宽度优先搜索。

\subsubsection{随机性策略}

\textbf{(5) 几何感知随机游走}:从高重要性点出发,按几何距离加权的随机游走生成序列。这引入探索性,防止过拟合固定模式。转移概率定义为:
\begin{equation}
P(i \to j) \propto \exp\left(-\frac{\|x_i - x_j\|^2}{2\sigma^2}\right) \cdot (s_j + \epsilon)
\end{equation}
$$e^{-\frac{d^2}{2\sigma^2}}$$
\subsection{几何感知的重要性场预测}
\label{sec:importance}

重要性场预测为动态路由提供几何先验。我们设计轻量级的重要性编码器,融合多尺度几何特征。

\subsubsection{局部几何描述符提取}

对于每个点$p_i$,在其$k$-近邻$\mathcal{N}_k(p_i)$上计算局部几何特征:

\textbf{(1) 曲率特征}:通过主成分分析(PCA)计算协方差矩阵$C_i$的特征值$\lambda_1 \geq \lambda_2 \geq \lambda_3$,定义:
\begin{equation}
\begin{aligned}
\text{linearity:} \quad & l_i = (\lambda_1 - \lambda_2) / \lambda_1 \\
\text{planarity:} \quad & p_i = (\lambda_2 - \lambda_3) / \lambda_1 \\
\text{scattering:} \quad & s_i = \lambda_3 / \lambda_1
\end{aligned}
\end{equation}

\textbf{(2) 法向变化}:法向量的局部方差$v_i = \text{Var}(\{n_j\}_{j \in \mathcal{N}_k(i)})$,反映表面平滑度。

\textbf{(3) 点云密度}:$d_i = k / V_i$,其中$V_i$为$k$-近邻的外接球体积。

组合几何描述符为$g_i = [l_i, p_i, s_i, v_i, d_i] \in \mathbb{R}^5$。

\subsubsection{多尺度重要性编码}

为了捕获不同尺度的几何模式,我们采用多尺度编码器:

\begin{equation}
\begin{aligned}
h_i^{(r)} &= \text{MLP}_r([f_i, g_i^{(r)}]), \quad r \in \{r_1, r_2, r_3\} \\
s_i &= \text{MLP}_{\text{agg}}([h_i^{(r_1)}, h_i^{(r_2)}, h_i^{(r_3)}])
\end{aligned}
\end{equation}

其中$r \in \{0.05, 0.1, 0.2\}$表示不同的邻域半径,$g_i^{(r)}$为对应尺度的几何描述符。最终重要性分数$s_i \in [0,1]$通过Sigmoid激活归一化。

\subsection{几何条件化动态路由网络}

动态路由网络是MASA的核心创新,负责根据局部几何特性为每个空间区域分配序列化权重。

\subsubsection{分块路由策略}

直接为每个点分配独立的路由权重会导致过度碎片化。我们采用\textbf{分块路由}(patch-wise routing):将点云划分为$M$个空间块$\{\mathcal{B}_1, \ldots, \mathcal{B}_M\}$(通过体素网格或FPS采样),每个块共享路由权重$w^{(m)} \in \Delta^K$。

\subsubsection{路由权重计算}

对于每个块$\mathcal{B}_m$,首先通过最大池化聚合块内特征:

\begin{equation}
\begin{aligned}
\bar{f}_m &= \text{MaxPool}(\{f_i\}_{i \in \mathcal{B}_m}) \\
\bar{g}_m &= \text{MeanPool}(\{g_i\}_{i \in \mathcal{B}_m}) \\
\bar{s}_m &= \text{MeanPool}(\{s_i\}_{i \in \mathcal{B}_m})
\end{aligned}
\end{equation}

然后通过门控网络计算路由权重:

\begin{equation}
\begin{aligned}
z_m &= \text{MLP}_{\text{gate}}([\bar{f}_m, \bar{g}_m, \bar{s}_m]) \in \mathbb{R}^K \\
w_m &= \text{Softmax}(z_m / \tau)
\end{aligned}
\end{equation}

其中$\tau$为温度参数,控制路由的sharp程度。训练初期使用较大的$\tau$(如1.0)保持探索性,训练后期退火至0.5使决策更加确定。

\subsubsection{软路由特征融合}

获得路由权重后,对$K$条路径的序列化特征进行加权融合。对于每个块$\mathcal{B}_m$内的点,首先按各序列化策略$\pi_k$重排并通过Mamba层:

\begin{equation}
h_i^{(k)} = \text{Mamba}_k(\{f_{\pi_k(j)}\}_{j=1}^{N})[i]
\end{equation}

其中$\text{Mamba}_k$表示针对第$k$种序列化训练的Mamba块(共享参数或专家参数),$h_i^{(k)}$为点$i$在第$k$条路径下的输出特征。

最终融合为:

\begin{equation}
h_i = \sum_{k=1}^K w_m^{(k)} h_i^{(k)}, \quad \forall i \in \mathcal{B}_m
\end{equation}

\subsubsection{负载均衡正则化}

为防止路由坍塌(所有块都选择同一路径),引入负载均衡损失:

\begin{equation}
\mathcal{L}_{\text{balance}} = \sum_{k=1}^K \left(\frac{1}{M}\sum_{m=1}^M w_m^{(k)} - \frac{1}{K}\right)^2
\end{equation}

鼓励各路径的平均使用率接近均匀分布。

\subsection{顺序感知选择性状态空间模型}

在Mamba块中,我们引入\textbf{序列化感知的选择性参数}:

\begin{equation}
\begin{aligned}
\Delta_i^{(k)} &= \text{softplus}(\text{Linear}_\Delta([f_i, e_k, s_i])) \\
B_i^{(k)} &= \text{Linear}_B([f_i, e_k]) \\
C_i^{(k)} &= \text{Linear}_C(f_i)
\end{aligned}
\end{equation}

其中$e_k \in \mathbb{R}^d$为第$k$种序列化策略的可学习嵌入(类似order prompt),使得Mamba能够感知当前使用的序列化类型并相应调整信息传播策略。例如,对于重要性优先序列化($k=3$),模型可以学习在序列早期(重要区域)使用较小的$\Delta$进行精细建模。

\subsection{训练策略}

\subsubsection{两阶段训练}

\textbf{阶段1:独立预训练}(Warm-up)。冻结路由网络,为每种序列化策略独立训练Mamba块,持续20 epochs:
\begin{equation}
\mathcal{L}_{\text{stage1}} = \mathcal{L}_{\text{task}}(\text{Mamba}_k(\mathcal{P}_{\pi_k})), \quad k = 1, \ldots, K
\end{equation}

这确保每条路径都学习到有效的特征表示,为后续路由学习提供良好初始化。

\textbf{阶段2:端到端联合训练}。解冻路由网络,端到端优化整个MASA,持续30 epochs:
\begin{equation}
\mathcal{L}_{\text{stage2}} = \mathcal{L}_{\text{task}} + \lambda_{\text{bal}} \mathcal{L}_{\text{balance}} + \lambda_{\text{reg}} \|\theta_{\text{router}}\|^2
\end{equation}

其中$\lambda_{\text{bal}} = 0.01$, $\lambda_{\text{reg}} = 0.0001$。

\subsubsection{路由温度退火}

采用cosine退火策略调整温度参数:
\begin{equation}
\tau(t) = \tau_{\min} + \frac{1}{2}(\tau_{\max} - \tau_{\min})(1 + \cos(\pi t / T))
\end{equation}

其中$\tau_{\max} = 1.0$, $\tau_{\min} = 0.3$, $T$为总训练步数。

\subsection{复杂度分析}

相比Pointer Network生成完整排列($O(N^2)$),MASA的时间复杂度为:

\begin{itemize}
\item 多路径序列化预计算:$O(KN\log N)$(排序操作)
\item 路由权重计算:$O(Md)$(块数$M \ll N$,特征维度$d$)
\item Mamba前向传播:$O(KNd)$(各路径并行)
\item 总复杂度:$O(KN\log N + KNd) = O(KN\log N)$
\end{itemize}

由于$K$为小常数,复杂度本质上为$O(N\log N)$,与单一固定序列化相当,远优于Pointer Network的$O(N^2)$。

\section{实验与分析}

\subsection{实验设置}

\subsubsection{数据集}

我们在三个基准数据集上进行评估:(1)\textbf{S3DIS}\cite{armeni20163d},室内场景语义分割数据集,包含6个大型区域,13类语义标签;(2)\textbf{ScanNet V2}\cite{dai2017scannet},室内场景分割数据集,包含1513个训练场景,20类物体;(3)\textbf{SemanticKITTI}\cite{behley2019semantickitti},室外激光雷达点云数据集,包含43552帧,19类道路场景。

\subsubsection{基线方法}

对比的基线方法包括:(1)\textbf{PTv3-Fixed},使用固定Z-order曲线的Point Transformer V3;(2)\textbf{PTv3-Hilbert},使用Hilbert曲线;(3)\textbf{PTv3-Random},每次训练随机打乱点顺序;(4)\textbf{PTv3-Multi},在不同层使用不同固定序列化(如\cite{wu2024point}中的4种顺序轮换)。

\subsubsection{实现细节}

重要性编码器采用3层PointNet++结构,隐藏维度128;路由网络为2层MLP,隐藏维度256。分块策略采用$0.5$m体素网格,平均每个场景生成$M \approx 200$个块。使用AdamW优化器,学习率$10^{-3}$,权重衰减$10^{-4}$。训练在4张V100 GPU上进行,batch size为8。

\subsection{主要结果}

表\ref{tab:masa_main}展示了MASA在三个数据集上的性能。MASA在所有数据集上均显著优于固定顺序基线:在S3DIS上达到\textbf{73.2\% mIoU},相比PTv3-Fixed提升\textbf{1.3\%};在ScanNet上达到\textbf{73.6\% mIoU},提升\textbf{1.2\%};在SemanticKITTI上达到\textbf{64.7\% mIoU},提升\textbf{1.5\%}。

\begin{table}[h]
\centering
\caption{不同序列化策略在语义分割任务上的性能对比}
\label{tab:masa_main}
\begin{tabular}{lccc}
\toprule
方法 & S3DIS (mIoU) & ScanNet (mIoU) & SemanticKITTI (mIoU) \\
\midrule
PTv3-Fixed (Z-order) & 71.9 & 72.4 & 63.2 \\
PTv3-Hilbert & 72.1 & 72.6 & 63.5 \\
PTv3-Random & 70.3 & 71.1 & 61.8 \\
PTv3-Multi & 72.3 & 72.8 & 63.8 \\
\midrule
\textbf{MASA (Ours)} & \textbf{73.2} & \textbf{73.6} & \textbf{64.7} \\
\ \ w/o dynamic routing & 72.5 & 72.9 & 63.9 \\
\ \ w/o importance & 72.7 & 73.1 & 64.2 \\
\ \ w/ hard routing & 72.8 & 73.2 & 64.3 \\
\bottomrule
\end{tabular}
\end{table}

\textbf{关键观察}:(1)随机顺序性能最差,说明序列化顺序至关重要;(2)Hilbert略优于Z-order,但差距不显著($<0.3\%$);(3)PTv3-Multi通过在不同层使用不同固定顺序获得一定增益,但仍无法适应具体场景;(4)MASA显著优于所有固定顺序,验证了动态路由的价值;(5)消融实验显示所有组件(动态路由、重要性预测)均有贡献。

\subsection{消融实验}

\subsubsection{序列化策略的贡献}

表\ref{tab:ablation_paths}分析了不同序列化策略组合的效果。结果表明:仅使用Z-order和Hilbert两种空间填充曲线已能达到72.6\%;加入重要性优先序列化进一步提升至72.9\%;完整的5种策略组合达到最佳性能73.2\%。

\begin{table}[h]
\centering
\caption{序列化策略组合消融实验(S3DIS数据集)}
\label{tab:ablation_paths}
\begin{tabular}{lc}
\toprule
策略组合 & mIoU (\%) \\
\midrule
仅Z-order (K=1) & 71.9 \\
Z-order + Hilbert (K=2) & 72.6 \\
+ 重要性优先 (K=3) & 72.9 \\
+ 密度自适应 (K=4) & 73.0 \\
+ 几何感知随机游走 (K=5,完整) & \textbf{73.2} \\
\bottomrule
\end{tabular}
\end{table}

\subsubsection{路由策略对比}

表\ref{tab:routing_strategies}对比了不同路由方式。硬路由(每个块只选择一条路径)虽然计算更高效,但性能略低0.4\%;均匀加权(所有路径权重相等)相当于简单集成,性能72.5\%;动态软路由取得最佳平衡。

\begin{table}[h]
\centering
\caption{路由策略对比}
\label{tab:routing_strategies}
\begin{tabular}{lccc}
\toprule
路由策略 & mIoU (\%) & 推理时间 (ms) & 内存 (GB) \\
\midrule
均匀加权 & 72.5 & 38 & 4.2 \\
硬路由 (Top-1) & 72.8 & 31 & 3.8 \\
软路由 (完整MASA) & \textbf{73.2} & 42 & 4.5 \\
\bottomrule
\end{tabular}
\end{table}

\subsubsection{分块粒度的影响}

图\ref{fig:block_size}展示了分块大小对性能的影响。过粗的分块($>1.0$m)导致路由粒度不足,无法适应局部几何变化;过细的分块($<0.2$m)则引入过多噪声且增加计算开销。我们发现$0.5$m为最佳平衡点。

\subsection{可视化分析}

\subsubsection{路由权重可视化}

图\ref{fig:routing_weights}展示了MASA在S3DIS场景中学习到的路由分布。观察发现:(1)\textbf{平面区域}(如地面、墙面)主要使用Z-order和Hilbert曲线,占比$>70\%$;(2)\textbf{复杂物体}(如桌椅)更多依赖重要性优先序列化(占比$\sim 45\%$);(3)\textbf{边缘和角点}展现出更均匀的路径分布,模型通过多路径融合捕获多角度信息。

\subsubsection{序列化路径对比}

图\ref{fig:scan_path_compare}对比了固定Z-order与MASA的动态路由在同一场景的序列化结果。固定Z-order按空间位置逐层扫描,可能先处理大片墙面和地板,后处理关键物体。而MASA在不同区域自适应选择:墙面使用高效的Z-order快速扫过;桌椅等物体切换至重要性优先,聚焦关键几何特征。

\subsubsection{重要性场与路由的关系}

图\ref{fig:importance_routing}展示了重要性分数与路由决策的相关性。高重要性区域(边缘、角点)倾向于选择重要性优先和几何感知随机游走,确保这些区域得到充分建模;低重要性区域则更多使用计算高效的空间填充曲线。

\subsection{泛化性与鲁棒性}

\subsubsection{跨数据集迁移}

在S3DIS上训练的MASA路由网络,直接应用于ScanNet(无需重新训练),仍能带来$+0.8\%$的提升(72.4\% $\to$ 73.2\%)。这说明学到的几何-路由映射具有一定泛化性,不同室内场景共享相似的几何模式。

\subsubsection{点云密度鲁棒性}

表\ref{tab:density_robust}展示了在不同采样率下的性能。固定顺序在低密度下性能下降明显(100\% $\to$ 25\%: $-3.5\%$),而MASA仅下降$1.2\%$。这是因为动态路由能够根据稀疏点云调整策略,如增加密度自适应序列化的权重。

\begin{table}[h]
\centering
\caption{不同点云密度下的性能}
\label{tab:density_robust}
\begin{tabular}{lccc}
\toprule
方法 & 100\%采样 & 50\%采样 & 25\%采样 \\
\midrule
PTv3-Fixed & 71.9 & 70.1 & 68.4 \\
MASA (Ours) & \textbf{73.2} & \textbf{72.4} & \textbf{72.0} \\
\bottomrule
\end{tabular}
\end{table}

\subsubsection{噪声鲁棒性}

我们在点云中添加高斯噪声($\sigma \in \{0.01, 0.02, 0.05\}$m)测试鲁棒性。MASA在$\sigma=0.05$时仍保持71.8\% mIoU,相比PTv3-Fixed的69.2\%显著更鲁棒。动态路由能够识别噪声导致的几何失真并调整序列化策略。

\subsection{计算效率分析}

表\ref{tab:efficiency}对比了不同方法的计算开销。MASA相比PTv3-Fixed增加约10\%的训练时间和8\%的推理时间,但获得1.3\%的性能提升,这是可接受的效率-性能权衡。

\begin{table}[h]
\centering
\caption{计算效率对比(S3DIS数据集,单场景)}
\label{tab:efficiency}
\begin{tabular}{lccc}
\toprule
方法 & 训练时间 (s/iter) & 推理时间 (ms) & GPU内存 (GB) \\
\midrule
PTv3-Fixed & 0.42 & 39 & 4.1 \\
PTv3-Multi & 0.45 & 41 & 4.3 \\
MASA (Ours) & 0.46 & 42 & 4.5 \\
\bottomrule
\end{tabular}
\end{table}

\section{本章小结}

本章提出了多路径自适应序列化网络(MASA),通过几何条件化的动态路由机制实现内容自适应的点云序列化。与直接学习任意排列的方法(如Pointer Network)相比,MASA采用多路径软路由的设计,在保持$O(N\log N)$线性复杂度的同时实现了自适应性,具有良好的实际可行性。

主要贡献包括:(1)提出基于多路径路由的自适应序列化框架,突破固定空间填充曲线的局限,同时避免了完整排列生成的组合爆炸问题;(2)设计5种互补的序列化策略,分别捕获空间局部性、几何重要性和探索性;(3)设计几何条件化的动态路由网络,通过分块软路由实现高效的端到端训练;(4)大规模实验验证了MASA在多个数据集上的有效性,并通过详细的消融实验和可视化分析揭示了动态路由的工作机制;(5)复杂度分析和效率实验证明了方法的实用性。

未来工作可以探索:(1)自动化的序列化策略搜索,通过神经架构搜索(NAS)发现更优的候选策略;(2)将MASA扩展到动态点云序列和4D时空建模;(3)结合多任务学习,为不同任务(分割、检测、配准)学习专用路由策略;(4)探索硬件加速的可能性,如CUDA kernel优化多路径并行处理。
