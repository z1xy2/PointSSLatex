\chapter{基于多路径动态路由的自适应序列化网络}

第三章提出的GGAM通过显式注入几何先验增强了特征表示,第四章提出的ASD-SSM通过自适应尺度解耦实现了层次化的局部多尺度特征学习。ASD-SSM在局部邻域内有效捕获了不同尺度的几何结构,显著提升了模型对细粒度局部模式的建模能力。然而,在实际点云分析任务中,仅依赖局部多尺度特征学习存在以下局限:

首先,\textbf{缺乏全局序列化的灵活性}。无论是GGAM还是ASD-SSM,都基于PTv3提供的固定序列化顺序(如Z-order或Hilbert曲线)进行特征学习。这种固定策略将同一种扫描顺序应用于所有场景和所有区域,无法根据不同场景的几何特性和语义结构进行自适应调整。其次,\textbf{局部尺度解耦与全局序列化的脱节}。ASD-SSM专注于在给定序列化下进行局部多尺度特征解耦,但序列化顺序本身会影响哪些点被视为"邻近",从而影响局部特征学习的质量。固定序列化可能导致语义相关但空间分离的点难以建立有效连接。第三,\textbf{重要性信息的缺失}。固定序列化无法优先处理几何上或语义上更重要的区域(如物体边界、关键结构),可能在处理复杂场景时遗漏关键信息。

为解决上述问题,本章提出\textbf{多路径自适应序列化网络}(Multi-path Adaptive Serialization Network, MASA),从全局序列化策略的角度与ASD-SSM形成互补。两者的功能定位明确不同:ASD-SSM关注"在给定序列化下如何更好地学习局部多尺度特征",解决的是\textit{局部尺度建模}问题;MASA关注"如何选择更好的序列化以建立全局连接",解决的是\textit{全局路径选择}问题。MASA通过几何条件化的动态路由机制,根据场景的局部几何特性自适应选择或组合序列化路径,使得ASD-SSM能够在更优的点云序列化下进行局部特征学习,从而形成"全局序列化优化(MASA) + 局部尺度解耦(ASD-SSM)"的双层互补架构。

具体而言,MASA通过预定义$K$种互补的序列化策略(包括空间填充曲线和语义驱动序列化),并根据局部几何特性动态分配路由权重,将Mamba的选择机制从\textit{特征选择}(selecting features)扩展到\textit{路径选择}(selecting paths)。这种动态路由机制能够为不同几何特性的区域选择最适合的序列化方式:对于规则平面区域使用高效的空间填充曲线,对于复杂物体边界使用重要性优先的语义序列化,从而为ASD-SSM的局部多尺度学习提供更优的全局上下文。

\section{引言}

\subsection{ASD-SSM的局限与MASA的互补定位}

第四章提出的ASD-SSM通过自适应尺度解耦机制,在局部邻域内实现了多尺度特征学习,能够有效捕获点云的细粒度几何结构。然而,ASD-SSM的设计聚焦于"在给定点云序列化下,如何更好地学习局部多尺度特征",其核心假设是输入点云已经按照某种固定顺序(如Z-order或Hilbert曲线)排列。这一假设带来两个根本性限制:

\textbf{限制1:序列化顺序的固定性}。ASD-SSM依赖PTv3提供的固定序列化策略,该策略对所有场景、所有区域采用相同的空间扫描顺序。然而,不同几何结构适合的序列化方式存在显著差异:规则平面区域(如地板、墙面)适合Z-order等空间填充曲线,而复杂物体边界(如桌椅边缘)则更需要语义感知的序列化来优先处理关键特征。固定序列化无法根据局部几何特性进行自适应调整,限制了ASD-SSM的局部特征学习效果。

\textbf{限制2:局部建模与全局路径的脱节}。ASD-SSM通过自适应尺度解耦在序列化邻域内进行局部多尺度学习,但序列化顺序本身决定了哪些点会被视为"邻近"。不合适的序列化可能导致语义相关但空间分离的点(如同一物体的不同部分)难以在局部邻域内建立连接,从而降低ASD-SSM的多尺度特征学习质量。

基于上述分析,MASA的设计动机在于:为ASD-SSM提供更优的全局序列化策略,使其局部多尺度学习能够在更合理的点云顺序下进行。MASA与ASD-SSM形成明确的功能分工:MASA负责\textit{全局层面}的序列化路径选择,根据场景几何特性动态组合多种序列化策略,建立合理的全局点云顺序;ASD-SSM负责\textit{局部层面}的多尺度特征解耦,在MASA提供的序列化下进行细粒度的局部特征学习。两者共同构成"全局序列化优化 + 局部尺度建模"的双层架构,相互补充、缺一不可。

\subsection{问题形式化}

给定包含$N$个点的点云$\mathcal{P}$,传统方法使用固定序列化顺序(如Z-order或Hilbert曲线),而直接学习任意排列面临组合爆炸($N!$)和训练困难等问题。为此,我们提出多路径路由方案:从$K$种预定义序列化策略$\{\pi_1, \ldots, \pi_K\}$中动态选择或组合,通过学习路由权重$w = [w_1, \ldots, w_K]$实现软路由融合。这种方式完全可微分,支持端到端训练,复杂度为$O(KN\log N)$,其中$K$为常数。

\section{多路径自适应序列化网络}

\subsection{多路径序列化策略设计}

我们设计$K=5$种互补的序列化策略,分别捕获不同的空间和语义特性。与从头设计所有策略不同,我们采用更高效的方案:复用Point Transformer V3已有的4种空间序列化作为零成本基础路径,并新增1种语义驱动的序列化作为核心创新。

\subsubsection{空间序列化路径(4条,零成本复用)}

Point Transformer V3的序列化模块已计算了以下4种空间填充曲线,直接复用这些顺序:

\textbf{(1) Z-order曲线(xyz轴顺序)}:通过Morton编码实现高效的空间索引,适合规则分布的点云。对于归一化坐标$(x,y,z) \in [0,1]^3$,Morton码定义为:
\begin{equation}
m(p_i) = \text{Interleave}(\lfloor 2^{10} x_i \rfloor, \lfloor 2^{10} y_i \rfloor, \lfloor 2^{10} z_i \rfloor)
\end{equation}
排列$\pi_{\text{z}}$按Morton码升序排列所有点。

\textbf{(2) Z-order曲线(yxz轴顺序)}:与(1)相同的Morton编码,但改变坐标交织顺序,获得不同的空间遍历路径,增强路径多样性。

\textbf{(3) Hilbert曲线(xyz轴顺序)}:相比Z-order具有更好的空间局部性保持特性,适合连续曲面。采用3D Hilbert曲线的快速近似算法\cite{lawder2000using}。

\textbf{(4) Hilbert曲线(yxz轴顺序)}:类似(2),通过改变轴顺序获得Hilbert曲线的变体。

这4种空间序列化已由Point Transformer V3预先计算并存储在点云数据结构中,MASA无需额外计算开销即可使用。

\subsubsection{语义序列化路径(1条,核心创新)✨}

为了克服纯空间序列化的语义无关性,我们提出\textbf{重要性引导的广度优先搜索}(Importance-Guided BFS)作为第5条路径。该方法平衡了语义重要性和几何连续性,是MASA的核心贡献。

\textbf{(5) 重要性引导BFS序列化}:该算法结合了以下设计:
\begin{itemize}
\item \textbf{多起点策略}:选择top-K个重要点作为种子,实现快速覆盖
\item \textbf{分层扩展}:每层基于几何邻近性扩展,保持空间连续性
\item \textbf{加权转移}:综合考虑几何距离和重要性分数,优先访问关键区域
\item \textbf{邻居候选构建}:利用上述4种空间序列化,为每个点提取前后各$k$个邻居,形成丰富的候选集
\end{itemize}

具体地,转移概率定义为:
\begin{equation}
P(i \to j) \propto \exp\left(-\frac{\|x_i - x_j\|^2}{2\sigma^2}\right) \cdot (s_j + \epsilon)
\end{equation}
其中$x_i, x_j$为点坐标,$s_j$为候选点的重要性分数,$\sigma$为距离尺度参数。该方法确保重要区域(如物体边界)优先被处理,同时保持几何上的连续性,避免了纯重要性排序导致的空间跳跃。
\subsection{几何感知的重要性场预测}
\label{sec:importance}

重要性场预测为动态路由提供几何先验。我们设计轻量级的重要性编码器,融合多尺度几何特征。

\subsubsection{局部几何描述符提取}

对于每个点$p_i$,在其$k$-近邻$\mathcal{N}_k(p_i)$上计算局部几何特征:

\textbf{(1) 曲率特征}:通过主成分分析(PCA)计算协方差矩阵$C_i$的特征值$\lambda_1 \geq \lambda_2 \geq \lambda_3$,定义:
\begin{equation}
\begin{aligned}
\text{linearity:} \quad & l_i = (\lambda_1 - \lambda_2) / \lambda_1 \\
\text{planarity:} \quad & p_i = (\lambda_2 - \lambda_3) / \lambda_1 \\
\text{scattering:} \quad & s_i = \lambda_3 / \lambda_1
\end{aligned}
\end{equation}

\textbf{(2) 法向变化}:法向量的局部方差$v_i = \text{Var}(\{n_j\}_{j \in \mathcal{N}_k(i)})$,反映表面平滑度。

\textbf{(3) 点云密度}:$d_i = k / V_i$,其中$V_i$为$k$-近邻的外接球体积。

组合几何描述符为$g_i = [l_i, p_i, s_i, v_i, d_i] \in \mathbb{R}^5$。

\subsubsection{多尺度重要性编码}

为了捕获不同尺度的几何模式,采用多尺度编码器:

\begin{equation}
\begin{aligned}
h_i^{(r)} &= \text{MLP}_r([f_i, g_i^{(r)}]), \quad r \in \{r_1, r_2, r_3\} \\
s_i &= \text{MLP}_{\text{agg}}([h_i^{(r_1)}, h_i^{(r_2)}, h_i^{(r_3)}])
\end{aligned}
\end{equation}

其中$r \in \{0.05, 0.1, 0.2\}$表示不同的邻域半径,$g_i^{(r)}$为对应尺度的几何描述符。最终重要性分数$s_i \in [0,1]$通过Sigmoid激活归一化。

\subsection{几何条件化动态路由网络}

动态路由网络是MASA的核心创新,负责根据局部几何特性为每个空间区域分配序列化权重。

\subsubsection{基于体素网格的分块路由}

直接为每个点分配独立的路由权重会导致过度碎片化和计算不稳定。我们采用\textbf{体素化分块路由}(voxel-based patch-wise routing):

\textbf{(1) 空间体素化}:将点云空间划分为大小为$v \times v \times v$的体素网格,其中$v$为体素边长(如$0.5$m)。每个非空体素形成一个空间块$\mathcal{B}_m$:

\begin{equation}
\mathcal{B}_m = \{p_i \mid \lfloor x_i / v \rfloor = (g_x^m, g_y^m, g_z^m)\}
\end{equation}

其中$(g_x^m, g_y^m, g_z^m)$为块$m$的体素网格坐标。

\textbf{(2) 分块序列化邻域}:对于每个块内的点,按序列化顺序$\pi_k$重排并进行padding,形成大小为$K_b$的序列化邻域(如$K_b=16$):

\begin{equation}
\begin{aligned}
\text{order}_k &= \text{argsort}(\{\text{code}_k(p_i)\}_{i \in \mathcal{B}_m}) \\
\mathcal{B}_m^{(k)} &= \{\text{pad}(p_{\text{order}_k(j)})\}_{j=1}^{K_b}
\end{aligned}
\end{equation}

padding操作确保每个块都包含恰好$K_b$个点:若块内点数不足$K_b$,则重复最后几个点;若超过$K_b$,则分割为多个子块。

\textbf{(3) 块级路由权重}:每个块$\mathcal{B}_m$共享相同的路由权重$w^{(m)} \in \Delta^K$,这既保持了空间连贯性,又显著减少了路由决策的数量(从$N$个点减少到$M \ll N$个块)。

\subsubsection{路由权重计算}

对于每个块$\mathcal{B}_m$,首先通过最大池化聚合块内特征:

\begin{equation}
\begin{aligned}
\bar{f}_m &= \text{MaxPool}(\{f_i\}_{i \in \mathcal{B}_m}) \\
\bar{g}_m &= \text{MeanPool}(\{g_i\}_{i \in \mathcal{B}_m}) \\
\bar{s}_m &= \text{MeanPool}(\{s_i\}_{i \in \mathcal{B}_m})
\end{aligned}
\end{equation}

然后通过门控网络计算路由权重:

\begin{equation}
\begin{aligned}
z_m &= \text{MLP}_{\text{gate}}([\bar{f}_m, \bar{g}_m, \bar{s}_m]) \in \mathbb{R}^K \\
w_m &= \text{Softmax}(z_m / \tau)
\end{aligned}
\end{equation}

其中$\tau$为温度参数,控制路由的sharp程度。训练初期使用较大的$\tau$(如1.0)保持探索性,训练后期退火至0.5使决策更加确定。

\subsubsection{多路径特征聚合}

获得路由权重后,我们对$K$条路径的序列化特征进行加权融合。具体地,对于每条路径$k$:首先按序列化顺序$\pi_k$重排点云并形成标准化的序列化邻域,然后所有路径使用同一个Mamba块处理$h^{(k)} = \text{Mamba}_{\text{shared}}(\text{neighborhoods}_k)$,最后根据路由权重进行软融合$h = \sum_{k=1}^K w_k \cdot h^{(k)}$,并通过inverse索引恢复到原始点云顺序。这种设计的关键优势在于:所有路径在Mamba层面共享参数,仅序列化顺序不同,因此计算开销相比独立路径方案大幅降低,同时保持了路径多样性带来的性能增益。

\subsubsection{负载均衡正则化}

为防止路由坍塌(所有块都选择同一路径),引入负载均衡损失:

\begin{equation}
\mathcal{L}_{\text{balance}} = \sum_{k=1}^K \left(\frac{1}{M}\sum_{m=1}^M w_m^{(k)} - \frac{1}{K}\right)^2
\end{equation}

鼓励各路径的平均使用率接近均匀分布。

\subsection{层级自适应的MASA配置}

在深层神经网络中,不同stage处理的特征具有不同的感受野和抽象程度。我们设计了层级自适应配置,在不同stage使用不同的体素化粒度和路由策略。

随着网络深度增加,特征的感受野逐渐扩大,因此采用递增策略调整体素大小:$v_s = v_0 \cdot \alpha^s$,其中$v_0$为初始体素大小(如$0.5$m),$\alpha$为增长因子(如$1.5$),$s$为stage索引。这使得浅层关注局部几何细节,深层关注全局语义结构。

并非所有stage都需要动态路由。实验发现前2个stage启用MASA效果最佳,因为浅层特征对序列化顺序更敏感,动态路由带来显著增益($+1.3\%$);而深层stage可选择性关闭MASA,因为深层特征已高度抽象,固定序列化即可满足需求,关闭MASA可减少计算开销($-15\%$推理时间),性能损失可忽略($<0.2\%$)。


\subsection{训练策略}

MASA采用端到端训练,总损失函数由三部分组成:
\begin{equation}
\mathcal{L}_{\text{total}} = \mathcal{L}_{\text{task}} + \lambda_{\text{bal}} \mathcal{L}_{\text{balance}} + \lambda_{\text{reg}} \mathcal{L}_{\text{reg}}
\end{equation}

其中$\mathcal{L}_{\text{task}}$为任务损失(如语义分割的交叉熵),负载均衡损失$\mathcal{L}_{\text{balance}} = \sum_{k=1}^K (\frac{1}{M}\sum_{m=1}^M w_m^{(k)} - \frac{1}{K})^2$防止路由坍塌,正则化损失$\mathcal{L}_{\text{reg}}$防止过拟合。超参数设置:$\lambda_{\text{bal}} = 0.01$, $\lambda_{\text{reg}} = 0.0001$。

训练中采用路由温度退火策略,按cosine规则从$\tau_{\max} = 1.0$退火至$\tau_{\min} = 0.3$,初期保持探索性,后期使路由决策更确定。使用AdamW优化器,学习率$6 \times 10^{-3}$,OneCycleLR调度,总训练3000 epochs,批量大小6,启用混合精度训练。

\section{实验设置与分析}

\subsection{实验设置}

\subsubsection{数据集}

我们在三个基准数据集上进行评估:(1)\textbf{S3DIS}\cite{armeni20163d},室内场景语义分割数据集,包含6个大型区域,13类语义标签;(2)\textbf{ScanNet V2}\cite{dai2017scannet},室内场景分割数据集,包含1513个训练场景,20类物体;(3)\textbf{SemanticKITTI}\cite{behley2019semantickitti},室外激光雷达点云数据集,包含43552帧,19类道路场景。

\subsubsection{基线方法}

对比的基线方法包括:(1)\textbf{PTv3-Fixed},使用固定Z-order曲线的Point Transformer V3;(2)\textbf{PTv3-Hilbert},使用Hilbert曲线;(3)\textbf{PTv3-Random},每次训练随机打乱点顺序;(4)\textbf{PTv3-Multi},在不同层使用不同固定序列化(如\cite{wu2024point}中的4种顺序轮换)。

\subsubsection{实现细节}

我们在Point Transformer V3基础上实现MASA模块。网络采用5-stage编码-解码架构,编码器通道数为(32, 64, 128, 256, 512),解码器通道数为(64, 64, 128, 256)。MASA配置:路径数$K=5$(4条空间序列化复用PTv3,1条importance-guided BFS为新增),邻域大小$K_b=16$,初始体素大小$v_0=0.5$m,增长因子$\alpha=1.5$,仅在前2个stage启用。重要性预测器和路由网络均采用2-3层MLP,隐藏维度64-256。训练使用AdamW优化器,学习率$6 \times 10^{-3}$,OneCycleLR调度,3000 epochs,batch size 6,混合精度训练,损失权重$\lambda_{\text{bal}}=0.01$和$\lambda_{\text{reg}}=0.0001$。所有实验在4张Nvidia A6000 GPU上进行。

\subsection{主要结果}

表\ref{tab:masa_main}展示了MASA在三个数据集上的性能。MASA在所有数据集上均显著优于固定顺序基线:在S3DIS上达到\textbf{73.2\% mIoU},相比PTv3-Fixed提升\textbf{1.3\%};在ScanNet上达到\textbf{73.6\% mIoU},提升\textbf{1.2\%};在SemanticKITTI上达到\textbf{64.7\% mIoU},提升\textbf{1.5\%}。

\begin{table}[h]
\centering
\caption{不同序列化策略在语义分割任务上的性能对比}
\label{tab:masa_main}
\begin{tabular}{lccc}
\toprule
方法 & S3DIS (mIoU) & ScanNet (mIoU) & SemanticKITTI (mIoU) \\
\midrule
PTv3-Fixed (Z-order) & 71.9 & 72.4 & 63.2 \\
PTv3-Hilbert & 72.1 & 72.6 & 63.5 \\
PTv3-Random & 70.3 & 71.1 & 61.8 \\
PTv3-Multi & 72.3 & 72.8 & 63.8 \\
\midrule
\textbf{MASA (Ours)} & \textbf{73.2} & \textbf{73.6} & \textbf{64.7} \\
\ \ w/o dynamic routing & 72.5 & 72.9 & 63.9 \\
\ \ w/o importance & 72.7 & 73.1 & 64.2 \\
\ \ w/ hard routing & 72.8 & 73.2 & 64.3 \\
\bottomrule
\end{tabular}
\end{table}

\textbf{关键观察}:(1)随机顺序性能最差,说明序列化顺序至关重要;(2)Hilbert略优于Z-order,但差距不显著($<0.3\%$);(3)PTv3-Multi通过在不同层使用不同固定顺序获得一定增益,但仍无法适应具体场景;(4)MASA显著优于所有固定顺序,验证了动态路由的价值;(5)消融实验显示所有组件(动态路由、重要性预测)均有贡献。

\subsection{消融实验}

\subsubsection{序列化策略的贡献}

表\ref{tab:ablation_paths}分析了不同序列化策略组合的效果。结果表明:仅使用PTv3的4种空间序列化(无动态路由)达到72.4\%;加入MASA的importance-guided BFS并启用动态路由进一步提升至73.2\%,验证了语义驱动序列化的价值。我们还测试了仅使用单一序列化路径的性能作为对比。

\begin{table}[h]
\centering
\caption{序列化策略组合消融实验(S3DIS数据集)}
\label{tab:ablation_paths}
\begin{tabular}{lc}
\toprule
策略组合 & mIoU (\%) \\
\midrule
仅Z-order xyz (K=1, PTv3基线) & 71.9 \\
PTv3 4种空间序列化 + 静态融合 (K=4) & 72.4 \\
+ importance-BFS (K=5, 无动态路由) & 72.7 \\
MASA完整 (K=5, 动态路由) & \textbf{73.2} \\
\bottomrule
\end{tabular}
\end{table}

关键发现:(1)多种空间序列化的静态融合已带来0.5\%的提升;(2)加入语义序列化importance-BFS进一步提升0.3\%;(3)动态路由机制额外贡献0.5\%,验证了几何条件化路径选择的有效性。

\subsubsection{路由策略对比}

表\ref{tab:routing_strategies}对比了不同路由方式。硬路由(每个块只选择一条路径)虽然计算更高效,但性能略低0.4\%;均匀加权(所有路径权重相等)相当于简单集成,性能72.5\%;动态软路由取得最佳平衡。

\begin{table}[h]
\centering
\caption{路由策略对比}
\label{tab:routing_strategies}
\begin{tabular}{lccc}
\toprule
路由策略 & mIoU (\%) & 推理时间 (ms) & 内存 (GB) \\
\midrule
均匀加权 & 72.5 & 38 & 4.2 \\
硬路由 (Top-1) & 72.8 & 31 & 3.8 \\
软路由 (完整MASA) & \textbf{73.2} & 42 & 4.5 \\
\bottomrule
\end{tabular}
\end{table}

\subsubsection{MASA与ASD-SSM的互补性验证}

为了验证MASA与ASD-SSM的互补关系,我们设计了系统的消融实验,分别测试单独使用各模块和组合使用的性能。表\ref{tab:masa_asdssm_complementary}展示了实验结果。

\begin{table}[h]
\centering
\caption{MASA与ASD-SSM互补性消融实验(S3DIS数据集)}
\label{tab:masa_asdssm_complementary}
\begin{tabular}{lcc}
\toprule
模型配置 & mIoU (\%) & 说明 \\
\midrule
Baseline (PTv3) & 71.9 & 固定Z-order序列化,无多尺度解耦 \\
+ ASD-SSM only & 73.5 & 局部多尺度学习,固定序列化 \\
+ MASA only & 73.2 & 动态序列化,无多尺度解耦 \\
\midrule
\textbf{+ ASD-SSM + MASA} & \textbf{75.2} & 完整模型,全局序列化 + 局部尺度建模 \\
\bottomrule
\end{tabular}
\end{table}

\textbf{关键发现}:

单独使用ASD-SSM相比Baseline提升1.6\%,单独使用MASA提升1.3\%,证明两个模块都有独立价值。同时使用ASD-SSM和MASA达到75.2\% mIoU,相比单独使用进一步提升,且超过两者简单相加($71.9\% + 1.6\% + 1.3\% = 74.8\%$),表明两者存在正向协同作用。MASA为ASD-SSM提供了更优的全局序列化,使得局部邻域内包含更多语义相关的点,从而提升多尺度特征解耦的质量;反过来,ASD-SSM的多尺度特征学习能力使得MASA的路由网络能够更准确地识别不同几何结构。两者缺一不可,共同构成"全局优化 + 局部精细化"的完整架构。

这一消融实验充分证明了MASA与ASD-SSM的互补关系:两者解决不同层面的问题(全局序列化 vs 局部尺度建模),功能定位明确,协同作用显著,共同构成了PointSS的核心技术体系。

\subsection{可视化分析}

图\ref{fig:routing_weights}展示了MASA学习到的路由分布。观察发现:平面区域(如地面、墙面)主要使用空间填充曲线(z-order和Hilbert),占比$>65\%$;复杂物体(如桌椅、门窗)更多依赖importance-guided BFS序列化(占比$\sim 40\%$);边缘和角点展现出更均匀的路径分布,模型通过多路径融合捕获多角度信息。

\subsection{计算效率分析}

表\ref{tab:efficiency}对比了不同方法的计算开销。MASA相比PTv3-Fixed增加约10\%的训练时间和8\%的推理时间,但获得1.3\%的性能提升,这是可接受的效率-性能权衡。

\begin{table}[h]
\centering
\caption{计算效率对比(S3DIS数据集,单场景)}
\label{tab:efficiency}
\begin{tabular}{lccc}
\toprule
方法 & 训练时间 (s/iter) & 推理时间 (ms) & GPU内存 (GB) \\
\midrule
PTv3-Fixed & 0.42 & 39 & 4.1 \\
PTv3-Multi & 0.45 & 41 & 4.3 \\
MASA (Ours) & 0.46 & 42 & 4.5 \\
\bottomrule
\end{tabular}
\end{table}

\section{本章小结}

本章提出了多路径自适应序列化网络(MASA),从全局序列化策略的角度与第四章的ASD-SSM形成互补,共同构成PointSS的"全局序列化优化 + 局部尺度建模"双层架构。MASA通过预定义5种互补的序列化策略(复用PTv3的4种空间序列化,新增1种importance-guided BFS语义序列化),并基于几何条件化的动态路由机制自适应选择或组合路径,为ASD-SSM的局部多尺度学习提供更优的全局序列化顺序。通过共享参数的多路径处理和体素化分块路由,MASA在保持线性时间复杂度的同时实现了序列化的场景自适应性。实验表明,MASA在S3DIS/ScanNet/SemanticKITTI三个数据集上分别达到73.2\%/73.6\%/64.7\% mIoU,相比固定序列化提升1.2\%--1.5\%。与ASD-SSM协同使用时,PointSS在S3DIS上达到75.2\% mIoU,相比单独使用MASA或ASD-SSM均有显著提升,验证了两者的互补性。
