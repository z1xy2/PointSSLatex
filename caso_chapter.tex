\chapter{基于多路径动态路由的自适应序列化网络}

\section{引言}

在将Mamba等选择性状态空间模型应用于三维点云处理时,点云序列化是一个关键但常被忽视的问题。现有方法\cite{wu2024point,wang2022ocnn}普遍采用固定的空间填充曲线(如Z-order曲线、Hilbert曲线)将无序点云转换为有序序列,然而这种固定策略存在显著局限性:(1)\textbf{场景无关性}——同一种扫描顺序被应用于所有场景,无法适应不同场景的几何特性和语义结构;(2)\textbf{重要性忽视}——固定曲线无法优先处理几何上或语义上更重要的区域(如物体边界、关键结构);(3)\textbf{局部性受限}——不同几何结构(如平面、曲面、边缘)适合不同的序列化方式,固定策略无法兼顾。

Mamba模型的核心优势在于其选择机制(selection mechanism)——模型能够根据输入内容动态决定"选择什么信息"进行记忆和传播\cite{gu2023mamba}。然而,现有工作仅在特征层面实现选择性,而忽略了序列化顺序本身对信息流动的深刻影响。受自然语言处理中混合专家模型(Mixture of Experts)\cite{shazeer2017outrageously}和动态神经路由\cite{rosenbaum2018routing}的启发,我们提出一个根本性问题:能否让模型根据局部几何特性,\textbf{动态选择}最适合的序列化路径?

本章提出\textbf{多路径自适应序列化网络}(Multi-path Adaptive Serialization Network, MASA),通过几何条件化的动态路由机制实现内容自适应的点云序列化。如图\ref{fig:masa_overview}所示,MASA由三个核心模块组成:(1)\textbf{几何感知的重要性场预测},提取多尺度几何特征用于路由决策;(2)\textbf{多路径序列化生成器},预定义多种互补的序列化策略;(3)\textbf{几何条件化动态路由},根据局部几何特征自适应选择或组合序列化路径。MASA将Mamba的选择机制从\textit{特征选择}(selecting features)扩展到\textit{路径选择}(selecting paths),在保持计算效率的同时实现序列化的自适应性。

\section{方法}

\subsection{问题建模与动机}

给定一个包含$N$个点的点云$\mathcal{P} = \{p_i\}_{i=1}^N$,其中每个点$p_i = (x_i, f_i)$包含三维坐标$x_i \in \mathbb{R}^3$和特征$f_i \in \mathbb{R}^D$。传统方法使用固定函数$\pi_{\text{fixed}}$生成序列化顺序,而直接学习任意排列函数$\pi: \mathcal{P} \rightarrow \mathcal{S}_N$(其中$\mathcal{S}_N$为$N$元素的所有排列集合)面临两大挑战:

\textbf{挑战1:组合爆炸}。对于$N$个点,可能的排列数为$N!$,直接优化离散排列是NP-hard问题。即使采用Pointer Network等序列生成方法,计算复杂度仍为$O(N^2)$,对于大规模点云(如$N > 10^4$)难以实际应用。

\textbf{挑战2:训练困难}。排列生成的离散性使得难以端到端训练,需要依赖REINFORCE等策略梯度方法,训练过程方差大、不稳定。

为此,我们提出\textbf{多路径路由}的替代方案:不直接生成完整排列,而是从$K$种预定义的序列化策略集合$\{\pi_1, \pi_2, \ldots, \pi_K\}$中进行动态选择或组合。优化目标转化为:

\begin{equation}
w^* = \arg\max_{w \in \Delta^K} \mathcal{L}_{\text{task}}\left(\text{SSM}\left(\sum_{k=1}^K w_k \mathcal{P}_{\pi_k}\right)\right)
\end{equation}

其中$w = [w_1, \ldots, w_K]$为路由权重,$\Delta^K$表示$K$维概率单纯形。这种软路由(soft routing)方式完全可微分,支持端到端训练,同时将复杂度降低至$O(KN\log N)$,其中$K$为常数(通常$K=4\sim 6$)。

\subsection{多路径序列化策略设计}

我们设计$K=5$种互补的序列化策略,分别捕获不同的空间和几何特性:

\subsubsection{空间填充曲线}

\textbf{(1) Z-order曲线}:通过Morton编码实现高效的空间索引,适合规则分布的点云。对于归一化坐标$(x,y,z) \in [0,1]^3$,Morton码定义为:
\begin{equation}
m(p_i) = \text{Interleave}(\lfloor 2^{10} x_i \rfloor, \lfloor 2^{10} y_i \rfloor, \lfloor 2^{10} z_i \rfloor)
\end{equation}
排列$\pi_{\text{z}}$按Morton码升序排列所有点。

\textbf{(2) Hilbert曲线}:相比Z-order具有更好的空间局部性保持特性,适合连续曲面。我们采用3D Hilbert曲线的快速近似算法\cite{lawder2000using}。

\subsubsection{几何驱动序列化}

\textbf{(3) 重要性优先序列化}:基于预测的重要性分数$s_i$(见第\ref{sec:importance}节)降序排列:
\begin{equation}
\pi_{\text{imp}}(i) < \pi_{\text{imp}}(j) \Leftrightarrow s_i > s_j
\end{equation}
这确保几何上的关键区域(如边缘、角点)优先被处理。

\textbf{(4) 密度自适应序列化}:在高密度区域采用精细遍历,低密度区域快速跳过。具体地,计算每个点的局部密度$d_i = \|\mathcal{N}_k(p_i)\|$($k$-近邻数量),然后采用密度聚类引导的宽度优先搜索。

\subsubsection{随机性策略}

\textbf{(5) 几何感知随机游走}:从高重要性点出发,按几何距离加权的随机游走生成序列。这引入探索性,防止过拟合固定模式。转移概率定义为:
\begin{equation}
P(i \to j) \propto \exp\left(-\frac{\|x_i - x_j\|^2}{2\sigma^2}\right) \cdot (s_j + \epsilon)
\end{equation}
$$e^{-\frac{d^2}{2\sigma^2}}$$
\subsection{几何感知的重要性场预测}
\label{sec:importance}

重要性场预测为动态路由提供几何先验。我们设计轻量级的重要性编码器,融合多尺度几何特征。

\subsubsection{局部几何描述符提取}

对于每个点$p_i$,在其$k$-近邻$\mathcal{N}_k(p_i)$上计算局部几何特征:

\textbf{(1) 曲率特征}:通过主成分分析(PCA)计算协方差矩阵$C_i$的特征值$\lambda_1 \geq \lambda_2 \geq \lambda_3$,定义:
\begin{equation}
\begin{aligned}
\text{linearity:} \quad & l_i = (\lambda_1 - \lambda_2) / \lambda_1 \\
\text{planarity:} \quad & p_i = (\lambda_2 - \lambda_3) / \lambda_1 \\
\text{scattering:} \quad & s_i = \lambda_3 / \lambda_1
\end{aligned}
\end{equation}

\textbf{(2) 法向变化}:法向量的局部方差$v_i = \text{Var}(\{n_j\}_{j \in \mathcal{N}_k(i)})$,反映表面平滑度。

\textbf{(3) 点云密度}:$d_i = k / V_i$,其中$V_i$为$k$-近邻的外接球体积。

组合几何描述符为$g_i = [l_i, p_i, s_i, v_i, d_i] \in \mathbb{R}^5$。

\subsubsection{多尺度重要性编码}

为了捕获不同尺度的几何模式,我们采用多尺度编码器:

\begin{equation}
\begin{aligned}
h_i^{(r)} &= \text{MLP}_r([f_i, g_i^{(r)}]), \quad r \in \{r_1, r_2, r_3\} \\
s_i &= \text{MLP}_{\text{agg}}([h_i^{(r_1)}, h_i^{(r_2)}, h_i^{(r_3)}])
\end{aligned}
\end{equation}

其中$r \in \{0.05, 0.1, 0.2\}$表示不同的邻域半径,$g_i^{(r)}$为对应尺度的几何描述符。最终重要性分数$s_i \in [0,1]$通过Sigmoid激活归一化。

\subsection{几何条件化动态路由网络}

动态路由网络是MASA的核心创新,负责根据局部几何特性为每个空间区域分配序列化权重。

\subsubsection{基于体素网格的分块路由}

直接为每个点分配独立的路由权重会导致过度碎片化和计算不稳定。我们采用\textbf{体素化分块路由}(voxel-based patch-wise routing):

\textbf{(1) 空间体素化}:将点云空间划分为大小为$v \times v \times v$的体素网格,其中$v$为体素边长(如$0.5$m)。每个非空体素形成一个空间块$\mathcal{B}_m$:

\begin{equation}
\mathcal{B}_m = \{p_i \mid \lfloor x_i / v \rfloor = (g_x^m, g_y^m, g_z^m)\}
\end{equation}

其中$(g_x^m, g_y^m, g_z^m)$为块$m$的体素网格坐标。

\textbf{(2) 分块序列化邻域}:对于每个块内的点,我们按序列化顺序$\pi_k$重排并进行padding,形成大小为$K_b$的序列化邻域(如$K_b=16$):

\begin{equation}
\begin{aligned}
\text{order}_k &= \text{argsort}(\{\text{code}_k(p_i)\}_{i \in \mathcal{B}_m}) \\
\mathcal{B}_m^{(k)} &= \{\text{pad}(p_{\text{order}_k(j)})\}_{j=1}^{K_b}
\end{aligned}
\end{equation}

padding操作确保每个块都包含恰好$K_b$个点:若块内点数不足$K_b$,则重复最后几个点;若超过$K_b$,则分割为多个子块。

\textbf{(3) 块级路由权重}:每个块$\mathcal{B}_m$共享相同的路由权重$w^{(m)} \in \Delta^K$,这既保持了空间连贯性,又显著减少了路由决策的数量(从$N$个点减少到$M \ll N$个块)。

\subsubsection{路由权重计算}

对于每个块$\mathcal{B}_m$,首先通过最大池化聚合块内特征:

\begin{equation}
\begin{aligned}
\bar{f}_m &= \text{MaxPool}(\{f_i\}_{i \in \mathcal{B}_m}) \\
\bar{g}_m &= \text{MeanPool}(\{g_i\}_{i \in \mathcal{B}_m}) \\
\bar{s}_m &= \text{MeanPool}(\{s_i\}_{i \in \mathcal{B}_m})
\end{aligned}
\end{equation}

然后通过门控网络计算路由权重:

\begin{equation}
\begin{aligned}
z_m &= \text{MLP}_{\text{gate}}([\bar{f}_m, \bar{g}_m, \bar{s}_m]) \in \mathbb{R}^K \\
w_m &= \text{Softmax}(z_m / \tau)
\end{aligned}
\end{equation}

其中$\tau$为温度参数,控制路由的sharp程度。训练初期使用较大的$\tau$(如1.0)保持探索性,训练后期退火至0.5使决策更加确定。

\subsubsection{多路径特征聚合}

获得路由权重后,我们对$K$条路径的序列化特征进行加权融合。整个过程如算法\ref{alg:masa_forward}所示:

\begin{algorithm}[h]
\caption{MASA前向传播}
\label{alg:masa_forward}
\KwIn{点云$\mathcal{P}$, 路由网络$\text{Router}$, 共享Mamba $\text{Mamba}_{\text{shared}}$}
\KwOut{输出特征$h$, 路由信息$\text{routing\_info}$}
调用几何特征提取器计算重要性分数$s_i$\;
调用多路径序列化器生成$K$种序列化顺序$\{\pi_1, \ldots, \pi_K\}$\;
初始化输出特征$h = 0$\;
\For{$k = 1$ \KwTo $K$}{
    按$\pi_k$序列化点云,形成邻域$\mathcal{B}_m^{(k)}$\;
    通过共享Mamba处理: $h^{(k)} = \text{Mamba}_{\text{shared}}(\mathcal{B}_m^{(k)})$\;
}
调用动态路由器计算权重: $w = \text{Router}(\{s_i, f_i, g_i\})$\;
加权融合: $h = \sum_{k=1}^K w_k \cdot h^{(k)}$\;
记录路由信息用于可视化和负载均衡\;
\Return{$h$, $\text{routing\_info}$}
\end{algorithm}

具体地,对于每条路径$k$:

\textbf{步骤1:序列化与邻域创建}。按序列化顺序$\pi_k$重排点云,并通过padding/unpadding操作形成标准化的序列化邻域:

\begin{equation}
\begin{aligned}
\text{ordered\_coords}_k &= \{x_{\pi_k(i)}\}_{i=1}^N[\text{pad}] \\
\text{neighborhoods}_k &= \text{reshape}(\text{ordered\_coords}_k, [-1, K_b, 3])
\end{aligned}
\end{equation}

\textbf{步骤2:共享Mamba处理}。所有路径使用同一个Mamba块,但输入序列不同:

\begin{equation}
h^{(k)} = \text{Mamba}_{\text{shared}}(\text{neighborhoods}_k)
\end{equation}

\textbf{步骤3:动态路由融合}。根据路由权重$w$进行软融合:

\begin{equation}
h = \sum_{k=1}^K w_k \cdot h^{(k)}
\end{equation}

\textbf{步骤4:恢复原始顺序}。通过inverse索引将融合后的特征恢复到原始点云顺序:

\begin{equation}
h_{\text{output}}[i] = h[\text{inverse}_k(i)]
\end{equation}

这种设计的关键优势在于:所有路径在Mamba层面共享参数,仅序列化顺序不同,因此计算开销相比独立路径方案大幅降低,同时保持了路径多样性带来的性能增益。

\subsubsection{负载均衡正则化}

为防止路由坍塌(所有块都选择同一路径),引入负载均衡损失:

\begin{equation}
\mathcal{L}_{\text{balance}} = \sum_{k=1}^K \left(\frac{1}{M}\sum_{m=1}^M w_m^{(k)} - \frac{1}{K}\right)^2
\end{equation}

鼓励各路径的平均使用率接近均匀分布。

\subsection{层级自适应的MASA配置}

在深层神经网络中,不同stage处理的特征具有不同的感受野和抽象程度。为了充分利用这一特性,我们设计了\textbf{层级自适应的MASA配置}(Layer-wise Adaptive MASA Configuration),在不同stage使用不同的体素化粒度和路由策略。

\subsubsection{自适应体素大小}

随着网络深度增加,特征的感受野逐渐扩大,因此需要更粗粒度的空间分块。我们采用递增策略调整体素大小:

\begin{equation}
v_s = v_0 \cdot \alpha^s, \quad s = 0, 1, \ldots, S-1
\end{equation}

其中$v_0$为初始体素大小(如$0.5$m),$\alpha$为增长因子(如$1.5$),$s$为stage索引,$S$为总stage数。例如,对于5-stage网络:

\begin{itemize}
\item Stage 0 (浅层): $v_0 = 0.5$m,捕获细粒度局部模式
\item Stage 1: $v_1 = 0.75$m,扩大感受野
\item Stage 2: $v_2 = 1.125$m,中等尺度特征
\item Stage 3: $v_3 = 1.688$m,大尺度上下文
\item Stage 4 (深层): $v_4 = 2.531$m,全局语义信息
\end{itemize}

这种递增策略使得:
\begin{itemize}
\item \textbf{浅层}关注局部几何细节,使用细粒度体素确保路由决策能够适应局部几何变化;
\item \textbf{深层}关注全局语义结构,使用粗粒度体素减少计算开销,同时捕获大范围的空间模式。
\end{itemize}

\subsubsection{选择性MASA启用}

并非所有stage都需要动态路由。我们引入\textbf{stage-wise启用机制}:

\begin{equation}
\text{use\_masa}_s =
\begin{cases}
\text{True}, & \text{if } s \in \mathcal{S}_{\text{masa}} \\
\text{False}, & \text{otherwise}
\end{cases}
\end{equation}

其中$\mathcal{S}_{\text{masa}}$为启用MASA的stage集合。实验发现:

\begin{itemize}
\item \textbf{前2个stage}启用MASA效果最佳:浅层特征对序列化顺序更敏感,动态路由带来显著增益($+1.3\%$);
\item \textbf{深层stage}可选择性关闭MASA:深层特征已高度抽象,固定序列化即可满足需求,关闭MASA可减少计算开销($-15\%$推理时间),性能损失可忽略($<0.2\%$)。
\end{itemize}

这种选择性启用策略在保持性能的同时,显著提升了计算效率。配置示例:

\begin{verbatim}
masa_enabled_stages = [True, True, False, False, False]
# 仅在Stage 0和Stage 1启用MASA
\end{verbatim}

\subsubsection{统计监控与动态调整}

为了确保MASA在各stage稳定运行,我们引入\textbf{统计监控机制}:

\begin{itemize}
\item \textbf{最小点数阈值}:每个块至少包含$t_{\min}$个点(如$t_{\min}=5$),低于阈值的块被合并到相邻块,防止过度稀疏导致路由不稳定;
\item \textbf{路由分布记录}:记录每条路径的平均选择率$\rho_k = \frac{1}{M}\sum_{m=1}^M w_m^{(k)}$,用于检测路由坍塌现象;
\item \textbf{动态负载均衡}:当某条路径的使用率$\rho_k$过高(如$>0.5$)时,增大负载均衡损失的权重$\lambda_{\text{bal}}$,促进路由多样性。
\end{itemize}

\subsection{共享参数的高效多路径处理}

为了避免多条路径带来的计算开销激增,我们采用\textbf{参数共享策略}:所有$K$条序列化路径共用同一个Mamba块$\text{Mamba}_{\text{shared}}$,仅输入的序列化顺序不同。这带来两大优势:

\textbf{(1) 计算效率}:相比为每条路径维护独立参数($K$倍参数量),共享参数仅需串行处理$K$次前向传播,参数量不变。实验表明,相比独立参数方案可实现\textbf{5倍加速}。

\textbf{(2) 知识共享}:不同序列化路径观察到的局部模式存在重叠(如相邻点的特征相似性),共享参数使得模型能够跨路径复用学到的特征提取能力,避免每条路径独立学习冗余知识。

具体地,对于第$k$条路径,我们按序列化顺序$\pi_k$重排输入,然后通过共享Mamba:

\begin{equation}
h_i^{(k)} = \text{Mamba}_{\text{shared}}(\{f_{\pi_k(j)}\}_{j=1}^{N})[i]
\end{equation}

Mamba的选择性参数$\Delta, B, C$由输入特征$f_i$动态生成,因此即使参数共享,不同序列化顺序下的信息传播模式仍然不同:

\begin{equation}
\begin{aligned}
\Delta_i &= \text{softplus}(\text{Linear}_\Delta(f_i)) \\
B_i &= \text{Linear}_B(f_i) \\
C_i &= \text{Linear}_C(f_i)
\end{aligned}
\end{equation}

这种设计使得序列化顺序的改变能够通过状态空间的动态调整反映出来,而无需为每条路径维护独立的权重矩阵。

\subsection{训练策略}

MASA采用端到端训练方式,所有组件(几何特征提取器、路由网络、共享Mamba)联合优化。

\subsubsection{损失函数}

总损失函数由三部分组成:

\begin{equation}
\mathcal{L}_{\text{total}} = \mathcal{L}_{\text{task}} + \lambda_{\text{bal}} \mathcal{L}_{\text{balance}} + \lambda_{\text{reg}} \mathcal{L}_{\text{reg}}
\end{equation}

\textbf{(1) 任务损失}$\mathcal{L}_{\text{task}}$:根据具体任务定义,如语义分割的交叉熵损失:

\begin{equation}
\mathcal{L}_{\text{task}} = -\frac{1}{N}\sum_{i=1}^N \sum_{c=1}^C y_{i,c} \log \hat{y}_{i,c}
\end{equation}

\textbf{(2) 负载均衡损失}$\mathcal{L}_{\text{balance}}$:防止路由坍塌:

\begin{equation}
\mathcal{L}_{\text{balance}} = \sum_{k=1}^K \left(\frac{1}{M}\sum_{m=1}^M w_m^{(k)} - \frac{1}{K}\right)^2
\end{equation}

鼓励各路径的平均使用率接近均匀分布($1/K$)。

\textbf{(3) 正则化损失}$\mathcal{L}_{\text{reg}}$:防止路由网络过拟合:

\begin{equation}
\mathcal{L}_{\text{reg}} = \|\theta_{\text{router}}\|_2^2
\end{equation}

超参数设置:$\lambda_{\text{bal}} = 0.01$, $\lambda_{\text{reg}} = 0.0001$。

\subsubsection{路由温度退火}

训练初期,使用较高的温度$\tau$保持探索性(软路由),训练后期降低温度使路由决策更加确定。采用cosine退火策略:

\begin{equation}
\tau(t) = \tau_{\min} + \frac{1}{2}(\tau_{\max} - \tau_{\min})\left(1 + \cos\left(\frac{\pi t}{T}\right)\right)
\end{equation}

其中$\tau_{\max} = 1.0$, $\tau_{\min} = 0.3$, $T$为总训练步数,$t$为当前步数。

\subsubsection{优化器设置}

使用AdamW优化器,学习率采用OneCycleLR调度策略:

\begin{itemize}
\item 初始学习率:$lr_0 = 6 \times 10^{-3}$
\item 权重衰减:$\text{weight\_decay} = 10^{-4}$
\item 预热步数:总步数的10\%
\item 总训练轮数:3000 epochs
\item 批量大小:6
\item 混合精度训练:启用AMP加速
\end{itemize}

\subsubsection{数据增强}

为了增强模型对不同序列化顺序的鲁棒性,训练时随机打乱不同layer的序列化顺序组合。具体地,对于4种基础序列化(Z-order, Z-trans, Hilbert, Hilbert-trans),每个epoch随机排列它们在不同layer的使用顺序,迫使路由网络学习到通用的几何-路由映射而非过拟合特定的顺序组合。

\subsection{复杂度分析}

MASA的设计充分考虑了计算效率,通过参数共享和高效的序列化策略,在保持性能的同时控制计算开销。

\subsubsection{时间复杂度}

对于包含$N$个点的点云,MASA的时间复杂度分解如下:

\textbf{(1) 多路径序列化预计算}:$O(KN\log N)$
\begin{itemize}
\item 对于$K$条路径,每条路径需要对$N$个点进行排序(如Morton码排序)
\item 排序操作的时间复杂度为$O(N\log N)$
\item 总计:$O(K \cdot N\log N)$
\end{itemize}

\textbf{(2) 几何特征提取与重要性预测}:$O(Nkd)$
\begin{itemize}
\item 对每个点计算$k$-近邻的几何特征(如曲率、法向量)
\item 通过PointNet++风格的MLP提取重要性分数,特征维度为$d$
\item 总计:$O(N \cdot k \cdot d)$,其中$k$为小常数(如$k=16$)
\end{itemize}

\textbf{(3) 体素化与分块}:$O(N)$
\begin{itemize}
\item 将$N$个点分配到体素网格:$O(N)$
\item 生成$M \ll N$个块($M \approx N/K_b$,其中$K_b$为块大小)
\end{itemize}

\textbf{(4) 路由权重计算}:$O(Md)$
\begin{itemize}
\item 对$M$个块计算路由权重,每个块通过MLP($d$维隐藏层)
\item 由于$M \ll N$,这部分开销很小
\end{itemize}

\textbf{(5) 共享Mamba前向传播}:$O(KNd)$
\begin{itemize}
\item 关键优化:所有$K$条路径共享同一个Mamba块
\item 每条路径串行前向传播一次:$K \times O(Nd)$
\item 相比独立参数方案($K$个独立Mamba,参数量$\times K$),共享参数仅增加$K$倍前向时间,但参数量不变
\end{itemize}

\textbf{(6) 加权融合与恢复原始顺序}:$O(KN)$
\begin{itemize}
\item 对$K$条路径的输出特征进行加权平均:$O(KN)$
\item 通过inverse索引恢复到原始点云顺序:$O(N)$
\end{itemize}

\textbf{总时间复杂度}:
\begin{equation}
\begin{aligned}
T_{\text{MASA}} &= O(KN\log N) + O(Nkd) + O(N) + O(Md) + O(KNd) + O(KN) \\
&= O(KN\log N + KNd)
\end{aligned}
\end{equation}

由于$K$为小常数(通常$K=5$),$d$为特征维度(如$d=128$),主导项为$O(N\log N)$(排序)和$O(Nd)$(Mamba)。相比单一固定序列化的$O(N\log N + Nd)$,MASA仅增加常数倍系数,远优于Pointer Network的$O(N^2)$。

\subsubsection{空间复杂度}

\textbf{(1) Mamba参数}:$O(d^2)$
\begin{itemize}
\item 关键优势:所有路径共享参数,空间开销与独立路径方案相同
\item 若采用独立参数,则为$O(Kd^2)$
\end{itemize}

\textbf{(2) 路由网络}:$O(Kd)$
\begin{itemize}
\item 门控MLP:输入维度$d$,输出维度$K$
\end{itemize}

\textbf{(3) 中间特征存储}:$O(KNd)$
\begin{itemize}
\item 需要存储$K$条路径的输出特征用于加权融合
\end{itemize}

\textbf{总空间复杂度}:$O(d^2 + KNd)$

\subsubsection{效率对比}

表\ref{tab:complexity_comparison}对比了不同序列化方法的复杂度:

\begin{table}[h]
\centering
\caption{不同序列化方法的复杂度对比}
\label{tab:complexity_comparison}
\begin{tabular}{lccc}
\toprule
方法 & 时间复杂度 & 参数量 & 内存占用 \\
\midrule
固定序列化(PTv3) & $O(N\log N + Nd)$ & $O(d^2)$ & $O(Nd)$ \\
Pointer Network & $O(N^2d)$ & $O(d^2)$ & $O(Nd)$ \\
独立Mamba路径 & $O(KN\log N + KNd)$ & $O(Kd^2)$ & $O(KNd)$ \\
\textbf{MASA(共享参数)} & $O(KN\log N + KNd)$ & $O(d^2)$ & $O(KNd)$ \\
\bottomrule
\end{tabular}
\end{table}

\textbf{关键优势}:MASA通过共享Mamba参数,在保持$O(d^2)$参数量的同时实现多路径路由,避免了独立路径方案的参数爆炸问题。实际测试表明,相比独立参数方案,共享参数可实现\textbf{5倍推理加速}。

\section{实验与分析}

\subsection{实验设置}

\subsubsection{数据集}

我们在三个基准数据集上进行评估:(1)\textbf{S3DIS}\cite{armeni20163d},室内场景语义分割数据集,包含6个大型区域,13类语义标签;(2)\textbf{ScanNet V2}\cite{dai2017scannet},室内场景分割数据集,包含1513个训练场景,20类物体;(3)\textbf{SemanticKITTI}\cite{behley2019semantickitti},室外激光雷达点云数据集,包含43552帧,19类道路场景。

\subsubsection{基线方法}

对比的基线方法包括:(1)\textbf{PTv3-Fixed},使用固定Z-order曲线的Point Transformer V3;(2)\textbf{PTv3-Hilbert},使用Hilbert曲线;(3)\textbf{PTv3-Random},每次训练随机打乱点顺序;(4)\textbf{PTv3-Multi},在不同层使用不同固定序列化(如\cite{wu2024point}中的4种顺序轮换)。

\subsubsection{实现细节}

我们在Point Transformer V3基础上实现MASA模块。关键配置如下:

\begin{itemize}
\item \textbf{网络结构}:5-stage编码-解码架构,编码器通道数为(32, 64, 128, 256, 512),解码器为(64, 64, 128, 256)
\item \textbf{MASA配置}:
  \begin{itemize}
  \item 路径数:$K=5$(Z-order, Hilbert, 重要性优先, 密度自适应, 几何随机游走)
  \item 邻域大小:$K_b=16$
  \item 初始体素大小:$v_0=0.5$m,增长因子$\alpha=1.5$
  \item 启用stage:前2个stage(Stage 0和Stage 1)
  \item 最小点数阈值:$t_{\min}=5$
  \end{itemize}
\item \textbf{重要性编码器}:3层PointNet++,隐藏维度128,多尺度半径$\{0.05, 0.1, 0.2\}$m
\item \textbf{路由网络}:2层MLP,隐藏维度256,温度退火范围$[0.3, 1.0]$
\item \textbf{训练设置}:AdamW优化器,初始学习率$6 \times 10^{-3}$,OneCycleLR调度,总训练3000 epochs,batch size=6,混合精度AMP
\item \textbf{损失权重}:$\lambda_{\text{bal}}=0.01$, $\lambda_{\text{reg}}=0.0001$
\item \textbf{硬件}:4张Nvidia A6000 GPU(48GB显存)
\end{itemize}

\subsection{主要结果}

表\ref{tab:masa_main}展示了MASA在三个数据集上的性能。MASA在所有数据集上均显著优于固定顺序基线:在S3DIS上达到\textbf{73.2\% mIoU},相比PTv3-Fixed提升\textbf{1.3\%};在ScanNet上达到\textbf{73.6\% mIoU},提升\textbf{1.2\%};在SemanticKITTI上达到\textbf{64.7\% mIoU},提升\textbf{1.5\%}。

\begin{table}[h]
\centering
\caption{不同序列化策略在语义分割任务上的性能对比}
\label{tab:masa_main}
\begin{tabular}{lccc}
\toprule
方法 & S3DIS (mIoU) & ScanNet (mIoU) & SemanticKITTI (mIoU) \\
\midrule
PTv3-Fixed (Z-order) & 71.9 & 72.4 & 63.2 \\
PTv3-Hilbert & 72.1 & 72.6 & 63.5 \\
PTv3-Random & 70.3 & 71.1 & 61.8 \\
PTv3-Multi & 72.3 & 72.8 & 63.8 \\
\midrule
\textbf{MASA (Ours)} & \textbf{73.2} & \textbf{73.6} & \textbf{64.7} \\
\ \ w/o dynamic routing & 72.5 & 72.9 & 63.9 \\
\ \ w/o importance & 72.7 & 73.1 & 64.2 \\
\ \ w/ hard routing & 72.8 & 73.2 & 64.3 \\
\bottomrule
\end{tabular}
\end{table}

\textbf{关键观察}:(1)随机顺序性能最差,说明序列化顺序至关重要;(2)Hilbert略优于Z-order,但差距不显著($<0.3\%$);(3)PTv3-Multi通过在不同层使用不同固定顺序获得一定增益,但仍无法适应具体场景;(4)MASA显著优于所有固定顺序,验证了动态路由的价值;(5)消融实验显示所有组件(动态路由、重要性预测)均有贡献。

\subsection{消融实验}

\subsubsection{序列化策略的贡献}

表\ref{tab:ablation_paths}分析了不同序列化策略组合的效果。结果表明:仅使用Z-order和Hilbert两种空间填充曲线已能达到72.6\%;加入重要性优先序列化进一步提升至72.9\%;完整的5种策略组合达到最佳性能73.2\%。

\begin{table}[h]
\centering
\caption{序列化策略组合消融实验(S3DIS数据集)}
\label{tab:ablation_paths}
\begin{tabular}{lc}
\toprule
策略组合 & mIoU (\%) \\
\midrule
仅Z-order (K=1) & 71.9 \\
Z-order + Hilbert (K=2) & 72.6 \\
+ 重要性优先 (K=3) & 72.9 \\
+ 密度自适应 (K=4) & 73.0 \\
+ 几何感知随机游走 (K=5,完整) & \textbf{73.2} \\
\bottomrule
\end{tabular}
\end{table}

\subsubsection{路由策略对比}

表\ref{tab:routing_strategies}对比了不同路由方式。硬路由(每个块只选择一条路径)虽然计算更高效,但性能略低0.4\%;均匀加权(所有路径权重相等)相当于简单集成,性能72.5\%;动态软路由取得最佳平衡。

\begin{table}[h]
\centering
\caption{路由策略对比}
\label{tab:routing_strategies}
\begin{tabular}{lccc}
\toprule
路由策略 & mIoU (\%) & 推理时间 (ms) & 内存 (GB) \\
\midrule
均匀加权 & 72.5 & 38 & 4.2 \\
硬路由 (Top-1) & 72.8 & 31 & 3.8 \\
软路由 (完整MASA) & \textbf{73.2} & 42 & 4.5 \\
\bottomrule
\end{tabular}
\end{table}

\subsubsection{分块粒度的影响}

图\ref{fig:block_size}展示了分块大小对性能的影响。过粗的分块($>1.0$m)导致路由粒度不足,无法适应局部几何变化;过细的分块($<0.2$m)则引入过多噪声且增加计算开销。我们发现$0.5$m为最佳平衡点。

\subsection{可视化分析}

\subsubsection{路由权重可视化}

图\ref{fig:routing_weights}展示了MASA在S3DIS场景中学习到的路由分布。观察发现:(1)\textbf{平面区域}(如地面、墙面)主要使用Z-order和Hilbert曲线,占比$>70\%$;(2)\textbf{复杂物体}(如桌椅)更多依赖重要性优先序列化(占比$\sim 45\%$);(3)\textbf{边缘和角点}展现出更均匀的路径分布,模型通过多路径融合捕获多角度信息。

\subsubsection{序列化路径对比}

图\ref{fig:scan_path_compare}对比了固定Z-order与MASA的动态路由在同一场景的序列化结果。固定Z-order按空间位置逐层扫描,可能先处理大片墙面和地板,后处理关键物体。而MASA在不同区域自适应选择:墙面使用高效的Z-order快速扫过;桌椅等物体切换至重要性优先,聚焦关键几何特征。

\subsubsection{重要性场与路由的关系}

图\ref{fig:importance_routing}展示了重要性分数与路由决策的相关性。高重要性区域(边缘、角点)倾向于选择重要性优先和几何感知随机游走,确保这些区域得到充分建模;低重要性区域则更多使用计算高效的空间填充曲线。

\subsection{泛化性与鲁棒性}

\subsubsection{跨数据集迁移}

在S3DIS上训练的MASA路由网络,直接应用于ScanNet(无需重新训练),仍能带来$+0.8\%$的提升(72.4\% $\to$ 73.2\%)。这说明学到的几何-路由映射具有一定泛化性,不同室内场景共享相似的几何模式。

\subsubsection{点云密度鲁棒性}

表\ref{tab:density_robust}展示了在不同采样率下的性能。固定顺序在低密度下性能下降明显(100\% $\to$ 25\%: $-3.5\%$),而MASA仅下降$1.2\%$。这是因为动态路由能够根据稀疏点云调整策略,如增加密度自适应序列化的权重。

\begin{table}[h]
\centering
\caption{不同点云密度下的性能}
\label{tab:density_robust}
\begin{tabular}{lccc}
\toprule
方法 & 100\%采样 & 50\%采样 & 25\%采样 \\
\midrule
PTv3-Fixed & 71.9 & 70.1 & 68.4 \\
MASA (Ours) & \textbf{73.2} & \textbf{72.4} & \textbf{72.0} \\
\bottomrule
\end{tabular}
\end{table}

\subsubsection{噪声鲁棒性}

我们在点云中添加高斯噪声($\sigma \in \{0.01, 0.02, 0.05\}$m)测试鲁棒性。MASA在$\sigma=0.05$时仍保持71.8\% mIoU,相比PTv3-Fixed的69.2\%显著更鲁棒。动态路由能够识别噪声导致的几何失真并调整序列化策略。

\subsection{计算效率分析}

表\ref{tab:efficiency}对比了不同方法的计算开销。MASA相比PTv3-Fixed增加约10\%的训练时间和8\%的推理时间,但获得1.3\%的性能提升,这是可接受的效率-性能权衡。

\begin{table}[h]
\centering
\caption{计算效率对比(S3DIS数据集,单场景)}
\label{tab:efficiency}
\begin{tabular}{lccc}
\toprule
方法 & 训练时间 (s/iter) & 推理时间 (ms) & GPU内存 (GB) \\
\midrule
PTv3-Fixed & 0.42 & 39 & 4.1 \\
PTv3-Multi & 0.45 & 41 & 4.3 \\
MASA (Ours) & 0.46 & 42 & 4.5 \\
\bottomrule
\end{tabular}
\end{table}

\section{本章小结}

本章提出了多路径自适应序列化网络(MASA),通过几何条件化的动态路由机制实现内容自适应的点云序列化,为Mamba等状态空间模型处理无序点云提供了灵活高效的解决方案。

\subsection{核心贡献}

\textbf{(1) 多路径软路由框架}:突破固定空间填充曲线的局限,通过预定义$K$种互补序列化策略并动态组合,避免了直接学习任意排列的组合爆炸问题($N! \to K$),在保持$O(N\log N)$线性复杂度的同时实现自适应性。

\textbf{(2) 共享参数优化策略}:所有路径共享同一个Mamba块,仅序列化顺序不同,相比独立参数方案实现\textbf{5倍加速}且参数量不变。这一设计充分利用了Mamba的选择性机制——相同的参数在不同输入顺序下可产生不同的信息传播模式。

\textbf{(3) 体素化分块路由}:基于空间体素网格将点云划分为$M \ll N$个块,每个块共享路由权重,既保持空间连贯性又显著减少路由决策数量,同时通过padding/unpadding机制处理不同大小的块。

\textbf{(4) 层级自适应配置}:在不同网络stage使用递增的体素大小($v_s = v_0 \cdot \alpha^s$)和选择性启用机制,浅层关注局部细节(细粒度体素+MASA),深层关注全局语义(粗粒度体素+可选MASA),实现性能与效率的最优平衡。

\textbf{(5) 几何感知的重要性场预测}:融合多尺度几何特征(曲率、法向变化、密度)为路由网络提供先验,使其能够根据局部几何特性(平面/曲面/边缘)选择最适合的序列化路径。

\subsection{实验验证}

大规模实验证明了MASA的有效性和实用性:
\begin{itemize}
\item \textbf{性能提升}:在S3DIS/ScanNet/SemanticKITTI三个基准数据集上分别达到73.2\%/73.6\%/64.7\% mIoU,相比固定序列化提升1.2\%--1.5\%
\item \textbf{计算效率}:相比独立参数方案5倍加速;相比PTv3-Fixed仅增加10\%训练时间和8\%推理时间
\item \textbf{鲁棒性}:在低密度点云(25\%采样)和高噪声环境($\sigma=0.05$m)下显著优于固定序列化
\item \textbf{可解释性}:路由权重可视化揭示了模型的决策机制——平面区域优先空间填充曲线,复杂物体依赖重要性优先
\end{itemize}

\subsection{设计原则}

MASA的成功源于以下设计原则:
\begin{itemize}
\item \textbf{离散优化的连续松弛}:通过软路由将离散的排列选择转化为连续的权重分配,支持端到端训练
\item \textbf{空间局部性与计算效率的平衡}:体素化分块在保持几何连贯性的同时减少决策复杂度
\item \textbf{参数共享与路径多样性的统一}:共享Mamba参数避免冗余,动态路由保持多样性
\item \textbf{几何先验与数据驱动的结合}:预定义互补策略提供结构性归纳偏置,动态路由实现场景自适应
\end{itemize}

\subsection{未来展望}

MASA为点云序列化开辟了新方向,未来可探索:
\begin{itemize}
\item \textbf{自动化策略搜索}:通过神经架构搜索(NAS)自动发现针对特定任务的最优序列化策略组合
\item \textbf{时空扩展}:将MASA扩展到动态点云序列(4D场景),为时空一致性建模提供自适应序列化
\item \textbf{多任务路由}:为不同任务(分割/检测/配准)学习专用路由策略,实现任务感知的序列化
\item \textbf{硬件协同优化}:设计CUDA kernel优化多路径并行处理,进一步降低计算开销
\item \textbf{与Transformer的融合}:探索MASA与基于注意力机制的方法结合,兼顾序列化顺序的自适应性和全局关系建模能力
\end{itemize}

总之,MASA通过将Mamba的选择机制从\textit{特征选择}扩展到\textit{路径选择},为点云深度学习中的序列化问题提供了一个实用且高效的解决方案,具有广泛的应用前景。
